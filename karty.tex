\chapter{Заводы на картах}

В эту главу я стащил сведения о кирпичных заводах, отмеченных на имеющихся у меня картах Киева, упорядочив всё по годам.\\

\noindent\textbf{1750} – на генплане Киева есть подпись «дорога на лыбедь хирпичным заводам» к дороге, идущей по крайней мере к лаврскому имению в низовьях Лыбеди, однако длинные сараи показаны и у противолежащего к южной Лысой горе (которая за Зверинцем) склона Бусовой горы, и около южного окончания Зверинецкого холма (юг ботсада), который там же плавно соединяется с Бусовой горой. По сути Бусова гора, Бусовица – это часть основного Зверинецкого холма, отделенная от него широким оврагом.\\

\noindent\textbf{1752} – на карте, снятой по Дукту дьяка Алферова спорных земель по иску Киево-Кирилловского монастыря с Киевским магистратом, на юго-восточном углу склона северной Лысой горы (нынешняя Юрковица, сам участок склона – срыт, уничтожен), примерно у нынешнего перекрестка улиц Кирилловской и Нижнеюрковской, показаны, уходящие по существовавшему тогда оврагу между Лысой и Щекавицей, на юго-запад, «кирпичные заводы Мещанина Ивана Григоровича, который речку Юрковицу принял на одну сажень».

\noindent\textbf{1753} – кирпичные заводы облепили всю долину между речкой Лыбедью и Бусовой горой, южную оконечность Зверинецкого холма, да берег между Лыбедью и Лысой горой около устья Лыбеди.\\


%\textbf{1833} – кирпичный завод на южном конце Труханова острова, между заливом Стариком и Днепром, напротив Аскольдовой могилы. Кирпичный завод на Черной горе примерно по месту завода Эйсмана (Субботиной), но дальше от дороги из Василькова, как если бы по месту будущего завода Шатовой. Вдоль завода течет ручей к прямоугольному пруду вероятно по месту озера Глинка.\\

\noindent\textbf{1838} – обозначены «Лаврские постройки кирпичных заводов» на левом берегу Лыбеди, рядом с ними ближе к склону Бусовой горы – «казенные кирп. заводы», и некий кирпичный завод на южном конце Зверинецкого холма (подпись неразборчивая).\\

\noindent\textbf{1842} – кирпичный завод Лавры, на Нижней Лыбеди. «Казенный кирпичный завод» вдоль подножия Бусовой горы. Другой казенный кирпичный завод – южнее хутора Корчеватое. Возле каждого из этих казенных заводов показаны арестантские казармы. Еще южнее вдоль Днепра, кирпичный завод между хутором Виноградным (ниже Китаевской пустыни) и Пирогово. Кирпичный завод вдоль южного конца Зверинецкого холма. Кирпичный завод у хутора Эйсмана. Кирпичный завод (сараи) на южном конце Труханова острова.\\

\noindent\textbf{1847} – кирпичные заводы купца Терехова на южной окончености Труханова острова (тот язык, что ближе к правому берегу).\\

\noindent\textbf{1860} – кирпичный завод на правом берегу Лыбеди в ее тогдашнем устье севернее Лысой горы.\\

\noindent\textbf{1863}, план Шуберта, вариант с железной дорогой. «Кирпичные сараи» к востоку от озера Рыбного. Севернее Мышеловки. Севернее Пирогово. На Сырце. Севернее хутора Грушечевичева (к востоку от «Преварки», севернее речки Сырца). На юго-восток от Броваров. На северо-восток от Броваров (на одной широте со станцией Бровары). Западнее Александровки, что близ Борисполя. На окраине Святильнова (ныне Светильня). Между Мокрецом и Заворичами. На северо-восточной окраине Остролучья. В Барышевке. В Вышгороде по берегу Днепра и южнее его, к востоку от урочища Волчьи горы, тоже на Днепре. Севернее Вышгорода в Балках, на берегу Днепра.\\

\noindent\textbf{1865}, завод у Кирилловской стоянки. Завод севернее Пирогово, но южнее Китаевской пустыни. «Казенные кирпичные заводы» между Корчеватым и Китаево. Казенные кирпичные заводы у Бусовой горы. Кирпичный завод у южного конца Зверинца.\\

\noindent\textbf{1879} – в Протасовом яру показаны два кирпичных завода, один без названия, а ближайший к железной дороге – «Батухина». Лаврский завод обозначен вероятно как «Красные сараи». Напротив Демиевки – «Кирпичный завод Шатовой», «Кирпичный завод Субботина». На углу Юрковской и переулка Юрковского, примерно там где сейчас кусок усадьбы по Нижнеюрковской, 2, ближе к восточной стороне военной части (Нижнеюрковская, 6) – «кафельный и кирпичный завод Гудим». На Сырце: заводы Рихерта (старшего), Марр, Серебренникова.\\

\noindent\textbf{1880} – обозначен, хотя не подписан точно, кирпичный завод Эйсмана-Субботиной.\\

\noindent\textbf{1886} – по Сырецкой улице, от Копыловской, против течения Сырца следуют заводы: Булышкина, Марр (ближе к Сырецкой), Булышкина (ближе к местности Шполянке), Рихерта, неизвестно чей, затем завод на хуторе Петерсона. По Большой Юрковской улице: Кафельный и кирпичный завод Гудимы. К югу напротив Демиевки: кирпичный завод Субботина. Протасов Яр, северный склон: некий кирпичный завод.\\

\noindent\textbf{1894} – Лаврский кирпичный завод на Теличке (на хуторе Лыбедском, он же Нижняя Лыбедь, дом 430), между железной дорогой и Лыбедью, южнее Военного (Госпитального) кладбища. Кирпичный завод Бернера в районе нынешней Лыбедской площади, между Большой Васильковской и Лыбедским переулком. В Протасовом яру, по северной стороне, показаны, от железной дороги последовательно по яру – кирпичный завод Батухина, кирпичный завод «чиновника Дзыгановского», обитавшего на Подоле. На Сырце показаны заводы: Марр на углу Копыловской и Сырецкой улиц, Булышкина непосредственно за Марр, но в сторону Шполянки, и еще один завод Марр сразу за первым Марр, выше по улице.\\

\noindent\textbf{1896} – кафельный завод на Демиевке, у перекрестка Кировоградской с Монтажников, по западную сторону от Монтажников. Лаврский завод между севером Лысой горы и Бусовой горой. У западного подножия Лысой горы – тоже кирпичный завод, возможно около пересечения Лысогорской с Набережным шоссе (тогда прямо к горе подходил Лысогорский рукав Днепра). Завод в Протасовом яру, северный склон.\\

\noindent\textbf{1899} – показан завод в тогдашнем устье Лыбеди, под Бусовой горой и ее Госпитальным кладбищем, между Лыбедью и железной дорогой. Вероятно это завод Лавры.\\

\noindent\textbf{1900} – завод Бернера на пересечении Большой Васильковской и Лыбедского переулка.\\

\noindent\textbf{1912} – кирпичный завод под Госпитальным кладбищем (между железной дорогой и Лыбедью), затем вдоль Черной горы против течения Лыбеди: кирпичный завод за Товарной станцией Киев II, завод, еще завод. Затем кирпичный завод (Бернера на Лыбедской). Завод по Брест-Литовскому шоссе по другую сторону Святошинской трамвайной станции, он же напротив западной части Святошино, восточнее железной дороги. Вероятно это к югу от станции метро Святошино. На другой карте того же года именно там – колония земледельческо-ремесленного приюта, а завод несколько восточнее, между хутором Галаганы и дачей Нивки, по тому же шоссе. В любом случае речь идет о нынешней промзоне между улицей Чистяковской и железной дорогой. Скорее всего это завод Д. Д. Байдашникова и братьев Фузиков, В. С. и С. С..\\

\noindent\textbf{1913} – Кирпичный завод к северо-западу от Иорданского переулка. Завод около будущего озера Глинки. Кирпичный завод Бернера на нынешней Лыбедской площади.\\

\noindent\textbf{1914} – завод Бернера. Завод у озера Глинки.\\

\noindent\textbf{1930-е} – на карте РККА видны: завод между станцией метро Лыбедская и железной дорогой, завод на юг от Глинки (ближе к Железнодорожному шоссе), завод чуть восточнее, завод над углом между ул. Киквидзе и Железнодорожным шоссе (собственно на изгибе Зверинецкой улицы, тогда грунтовой дороги по плато горы), завод у перекрестка улиц Олега Кошевого и Цымбалова яра, заводы между Лысой горой и Мышеловкой, завод на Татарской (бывший Зайцева), завод на углу Фрунзе и Нижнеюрковской, завод на месте военной части на Нижнеюрковской, завод возле нынешнего адреса Копыловская, 67 К 11, завод около Копыловской, 38, завод через улицу напротив Сырецкой, 42/44, завод на юг от  Сырецкой 31, завод около Сырецкой, 33-Ш, завод на Козелецкой улице на хуторе Грушки (близ железной дороги), завод напротив центрального вокзала (через пути).\\

\noindent\textbf{1941} – завод около озера Глинки.\\

\noindent\textbf{1943} – завод у низовий речки Бусловки, между холмом и железной дорогой, по восточную сторону улицы Киквидзе. Два завода на юг от Лысой горы, затем один еще южнее у Мышеловки. Три кирпичных завода один за другим вдоль восточного склона Лысой горы, до Мышеловки. Кафельный или кирпичный завод у речки Совки на пересечении Кировоградской с Монтажников. Завод на северном склоне Протасова Яра.

\chapter*{Из Киевлянина №254, 1898}
\addcontentsline{toc}{chapter}{Из Киевлянина №254, 1898}

\textbf{Кирпичные заводы Киева и окрестностей, часть IV}

Из кирпичных заводов долины Лыбеди в самых благоприятных условиях относительно сбыта кирпича находятся заводы Я.Н. Бернера, В.А. Субботина и наследников г-жи Шатовой. Так как они расположены в конце Большой Васильковской улицы, то при доставке кирпича покупателям пользуются городскою мостовую. Поэтому, в то время как другие заводы, находящиеся вне городской черты, принуждены во время весенней и осенней распутицы приостанавливать на некоторое время доставку кирпича в город, распутица не причиняет почти никаких хлопот трем вышеупомянутым заводам.

Завод Я.Н. Бернера имеет 12 столов и 1 Гофманскую печь в 16 камер, вместимостью в 200 000 штук кирпича. Заводу принадлежит около 9 десятин земли и глинище его находится весьма близко от сараев, в которых производится выделка кирпича. Завод вырабатывает в год 4 1/2 – 5 млн. кирпича и на нем работает до 100 чел. рабочих и свыше 60 лошадей, доставляющих глину к милкам и месящих ее.

Завод В. Субботина по размерам производства почти в полтора раза превышает завод Я.Н. Бернера и его производительность превышает 6 млн. штук кирпича ежегодно. Заводу принадлежит обширная площадь земли и, несмотря на то, что принадлежащий ему пласт глины эксплоатируется уже более 60 лет, до сих пор разрыта лишь ничтожная часть территории завода. Заводу приходится производить довольно значительную съемку земли и его глинище представляет собой довольно глубокий овраг. На заводе имеются две гофманские печи. Количество рабочих превышает 130 человек и, кроме того, на заводе работает около 80 подвод. 

Завод В.А. Субботина относится к числу самых благоустроенных кирпичных заводов в окрестностях Киева. Он имеет массу древесных насаждений и весь утопает в зелени. Для рабочих устроены специальные казармы, представляющие собой капитальное каменное здание. На других заводах помещения для рабочих отличаются чрезмерной простотой и представляют собой часто ряд отдельных небольших каморок, приютившихся, наподобие ласточкиных гнезд, под навесом гофманской печи. Сами рабочие мало разборчивы на счет помещения\footnote{Ну да, как же. Что хозяин завода предоставит, там и живут.}, ютятся часто в сараях, в пещерах, спят прямо под открытым небом.

Завод наследников г-жи Шатовой помещается рядом с заводом В.А. Субботина и его усадьба представляет собой узкую возвышенную полосу земли, тянущуюся вдоль Васильковской дороги. Производительность завода не велика, благодаря тесноте места, и он выделывает всего не более 2 млн. кирпича в год. Гофманская печь рассчитана на 80 000 штук кирпича. Число рабочих не превышает 50 человек.

Три вышеописанных хавода расположены по соседству друг от друга в конце Большой Васильковской улицы. Следующий завод долины Лыбеди, Лаврский, находится в версте от них вниз по Лыбеди. Сам завод, гофманская печь и сараи для выделки и сушки кирпича помещаются по правую сторону полотна Московско-Киево-Воронежской дороги, его глинище по левую сторону полотна, в 200 саж. от завода. Выделка кирпича производится на семи столах и составляет около 2 млн. штук ежегодно. Число рабочих превышает 60 человек.

Следуя вниз по течению Лыбеди, которая после выхода из лаврского пруда имеет крайне жалкий вид, мы попадаем на большой завод, устроенный на земле Киево-Выдубецкого монастыря и принадлежащий Я.Н. Бернеру. Прежним арендатором, Эрлихом, производилась здесь машинная выделка кирпича, в воспоминание о чем сохранилась высокая двухэтажная постройка, в которой теперь помещаются лошади и коровы. Так как завод расположен у самого подножья Лысой горы с ее фортами, то ему не разрешается возводить постоянных построек, в виду чего все сараи для просушки кирпича выстроены из самого дешеваго материала и носят временный характер.

На заводе имеются две гофманские печи, вместимостью по 200 000 кирпича каждая. Глинище расположено вблизи от центра завод аи при том оно довольно мелко, благодаря чему доставка глины в малки не представляет никаких затруднений. Благодаря этому на заводе сравнительно меньше лошадей, чем на других, и число их не превышает 50. На заводе работает до 150 человек рабочих. Он производит 7 млн. кирпича.

Завод Я.Н. Бернера отделен лишь межою от бывших казенных заводов, принадлежащих в настоящее время наследникам Снежко, которым принадлежат, вообще, около 100 десятин земли, занятой под заводские постройки вплоть до самого Китаево, или Мышеловки, к которой приписаны в административном отношении заводы. Три завода Снежко вырабатывают в общей сложности около 14 млн. кирпича. На заводах устроены четыре гофманские печи, вместимостью в 200 000 штук кирпича каждая. Выделка кирпича производится на 40 столах и число рабочих превышает 340 человек. Два завода арендуются у наследников Снежко, Фокиным и Доломашкиным, третий, самый большой, с двумя гофманскими печами, эксплоатируется непосредственно самими владельцами, при чем все годовое производство продано вперед Л. Козинскому. Завод принадлежит к числу более благоустроенных, все сараи построены из хорошего материала и содержатся в порядке.

Вторая группа кирпичных заводов, расположенных в окрестностях Киева, помещаются в долине безымянного ручья, вытекающего из пруда Голосеевской пустыни. Голосеевский пруд занимает центральную часть котловины, окруженной горами, покрытыми лесом, по преимуществу дубом, и питается ключами, берущими начало и подножья гор. Голосеевская роща, принадлежащая Киево-Печерской лавре, и Голосеевский пруд недоступны публике и малоизвестны в Киеве, в то время как по красоте положения редкая местность в окрестностях Киева может соперничать с Голосеевым.

Из этого живописнейшего озера вытекает небольшой ручеек, вливающийся в Днепр почти в том же пункте, что и Лыбедь. В прежние годы ручеек был, вероятно более многоводным и на нем стояло несколько мельниц, ни одна из которых не работает теперь за отсутствие воды в ручье, которые еще мельче Лыбеди. Долина ручья гораздо уже, чем долина Лыбеди, и берега ея гораздо круче. Вся долина на протяжении от Голосеевского леса до Китаево, или собственно Мышеловки, занята кирпичными заводами. Сараи для выделки и сушки кирпича, гофманские печи и прочие строения покрывают сплошь всю долину, глинища же их врезываются в горы берегов долины. Занимая собой узкое ущелье, заводы сильно стеснены и им расширяться уже некуда, так что приходится утилизировать каждую квадратную сажень долины ручья.

Последний представляет уже в настоящее время узкую канаву, так как болотистые, низкие берега его подсыпаются заводами и на полученном таким образом насыпном грунте возводятся все новые и новые постройки. Еще несколько лет тому назад ручей в одном пункте был запружен и образовывал второй пруд, теперь этот пруд уже почти совершенно засыпан.

Земли, расположенные по берегам ручья, принадлежат крестьянам с. Мышеловки, или ведомству государственных имуществ, у которых арендуются владельцами кирпичных заводов. В долине Голосеевского ручья расположены следующие заводы: Д.Л. Горчакова, И.С. Ясько, А.К. Рейхе, Калашникова и Лунева. В отношении эксплоатации эта группа заводов представляет то неудобство, что соединены с городом крайне неудобною, гористою дорогою, которая осенью и весной превращается в сплошное болото.

Завод Д.Л. Герчикова расположен в верхней части долины ручья на границе с дачею лавры. Выделка кирпича производится на 12 столах, число рабочих 110-120 человек, гофманская печь одна на 200 000 кирпича. Завод производит до 4 1/2 млн. кирпича.

Завод И.С. Ясько несколько меньше предыдущего и выделка кирпича производится на нем на 10 столах. Гофманская печь одна, число рабочих простирается до 100 человек. Завод производит около 3 млн. кирпича.

Завод А.К. Рейхе располагает столь незначительной площадью, что не только вся его усадьба застроена сараями, но даже пришлось утилизировать чердачное помещение над гофманской печью. Это помещение несколько повышено и в нем установлены два стола для выделки кирпича. Так как гофманская печь построена у подножия горы, то подавать кирпич в чердачное помещение не представляет особых затруднений. На заводе имеется 12 столов и его годичное производство превышает 4 1/2 млн. Число рабочих 120.

Завод Калашникова принадлежит к числу самых маленьких заводов в окрестностях Киева. Выделка кирпича производится здесь на четырех столах. На заводе 30 человек рабочих, 10 лошадей, одна гофманская печь, вместимостью 150 000 кирпича, которого завод обжигает, вообще, около 1 1/2 млн. штук.
   
На последнем заводе этой группы, Лунева, выделка кирпича производится на 11 столах. На заводе работают 85 человек и завод производит около 3 1/2 млн. кирпича.

В.Ц.
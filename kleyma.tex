\chapter{Клейма}

Тут помещаю известные мне клейма с предложением их толкования, а иногда без него. Всё что относится к казенным заводам и Лавре смотрите в главе «Производители».

Если у одного и того же производителя адреса разнятся – то указана улица Кирилловская, то один из вариантов Юрковской – это скорее всего означает, что завод занимал две смежные усадьбы, при этом нижняя находилась под горой на улице Кирилловской, а верхняя на горе, по улице такой-то Юрковской. 

С Юрковской, или Большой Юрковской улицей со второй половины 19 века начинается страшная путаница. Части этой улицы в разное время известны под разными названиями и порой входят в ныне в состав других улиц. Юрковская распадалась на Юрковскую, Нижнеюрковскую, Верхнеюрковскую, частично Багговутовскую, Подлесную, Нагорную, Шмидта. Кроме того, существовала еще Мало-Юрковская. Изменение связи этих названий с дорогами на протяжении времени – отдельный вопрос.

Про Кирилловскую улицу (Фрунзе) – в 1898 году часть номеров усадеб сдвинулась (например, было 69, стало 71). Я стараюсь оговаривать эти сдвиги для кирпичных заводов, если мне о них известно и сдвиг произошел для конкретной усадьбы.

В источниках фамилии производителей и арендаторов иногда разнятся. Я привожу данные, как они указаны в источнике.

Согласно исследованиям Валерия Мельникова законов Российской империи относительно клеймления кирпичей, клеймление за 19 век обуславливали по крайней мере три указа. 

Указ №3467 от 5 февраля 1830 предписывал помещать на клейме имя и фамилию производителя «хотя начальными буквами, и места, где находится». Клеймо предлагалось регистрировать в Департаменте мануфактур и внешней торговли. Кирпичи с клеймами получали некоторые преимущества при поставках за границу. 

Указом №20848 от 24 января 1847 «О мерах для прочной и правильной выделке кирпича» вводилось обязательное клеймление, без уточнения содержимого клейма. Вероятно, содержимое определялось указом 3467.

Наконец, законом №12553 от 26 февраля 1896 года клейма причисляются к товарным знаками, регистрируются в Департаменте Торговли и Мануфактур (посредством Департамента прошения). При этом товарные знаки, которые не зарегистрированы таким образом, «должны содержать в себе обозначение, на Русском языке: 1) имени и отчества владельца торгового или промышленного предприятия (хотя бы инициалами), а также его фамилии или наименования фирмы (полностью), и 2) местонахождения предприятия». 

Иными словами, какой-нибудь вензель, монограмма прочие виды кратких клейм становились торговыми знаками и должны были регистрироваться, за уплату определенной пошлины.\\

\noindent\textbf{А.Б.} – Александра Даниловна Булышкина (Булыжкина), купчиха, вторая жена купца-старовера 2-й гильдии Тимофея Кононовича Булышкина (1818-189х). Завод был основан на Куреневке в 1852 году отцом Тимофея, Кононом Маркеловичем\footnote{В 1847 году обоих Булышкиных, старшего и младшего, преследовали за общение с заграничными старообрядцами. Из доклада 5 мая 1847 г. министра Перовского: 

\begin{quotation}
в конце того же мая задержаны и доставлены в С.Петербург Геронтий Леонов и Абрам Ушаков и прикосновенные к ним: иностранец Австрийский подданный Иоганн Миллер, Киевский 2 гильдии купец Конон Булышкин, сын его Тимофей [...]

Об лицах этих по высочайшему повелению произведено действит. стат. советником Липранди строжайшее исследование, под непосредственным наблюдением шефа жандармов и министра внутренних дел. По исследованию сему названные выше лица оказались виновными: [...] Бульшкин отец, в посредстве сношений заграничных раскольников с нашими, устроением у себя в доме, в Киеве, перепутья проникающим в Россию раскольникам и сосредоточия, куда со всех сторон России стекались письма и посылки за границу. 

[...] За таковые проступки, по Высочайшему повелению, последовавшему в 27 день июня 1847 года на всеподданейшем докладе шефа жандармов и министра внутренних дел, подвергнуты: [...] Булышкины – взысканию отосланных чрез них за границу 1 442 руб., строжайшему надзору местного начальства и обязанием подпискою, что впредь не будут участвовать в сношениях с раскольниками. 
\end{quotation}

В 1861 году, когда на Днепре случилось большое наводнение, Конон Булышкин отдал свой дом десяти пострадавшим семьям с Оболони.}, тоже купцом 2-й гильдии. Удаляясь в 1865 году от мiра в Чернобыльский Пустынно-Никольский монастырь\footnote{Вероятно, именно он прослыл там как инок Корнилий. При себе в монастыре сей инок всё же кое-что имел – ценные бумаги, векселя, и передал их работавшему в монастыре, будущему киевскому купцу Олейникову Даниилу Андриановичу в качестве начального капитала.}, Конон передал все дела и имущество сыну, о чем свидетельствует запись Киевской палаты гражданского суда:

\begin{quotation}
По старости лет моих и совершенной слабости моего здоровья, сам я заниматься своими делами не имею уже возможности, а потому уполнамачиваю тебя, любезный сын мой, вместо себя, и поручаю тебе все свои дела, как по управлению всем моим недвижимым имуществом, заключающемся в состоящих Плоской г. Киева части в 1 квартале с пристройками, а второе под № 16 каменный одноэтажный дом также с пристройками, и в предместье г. Киева Куреневском квартале кирпичном заводе.
\end{quotation}

Кирпичный завод этот состоял из 7 сараев, помещения для обжигательных печей, двух деревянных домов и глинища.

Рядом были куплены еще четыре усадьбы, вероятно для расширения производства. По сведениям на 1882 год, завод занимал усадьбы по адресам Копыловская, 26 и 28 – надо полагать, вобравшие в себя приобретенные четыре.

С конце 1890-х завод, после смерти Тимофея Кононовича, перешел к его супруге. 

На 1887 завод произвел 1 600 000 кирпичей, годовой оборот\footnote{Вырученные за год средства. Если отнять от них расходы за год (стоимость сырья, зарплату рабочим и так далее), получим, сколько прибыли.} составил 26 000 рублей, при 80 рабочих. 1890: оборот 24 000 рублей. 1894: 1 500 000 кирпичей, оборот 27 000 рублей, 72 или 60 рабочих, заведующий купец Федор Ник. Поддубный\footnote{В 1902 году Поддубный то ли владел, то ли арендовал завод в с. Мостище около Ирпеня, этот завод приостановил работу в 1902 году.}. 1900: 95 рабочих (24 местных), управляющий Федор Раузен. 

Александра Даниловна Булышкина упомянута владелицей завода еще в 1903 и 1910 годах. 

На 1910-й, для завода по Копыловской, 61, данные таковы: заведующий Козинский Я. Б., 4 000 000 кирпичей, оборот 52 000 рублей, 90-100 рабочих. См. «Гусева Екатерина Гавриловна» в главе о производителях – ей в это время принадлежал завод.

На 1915 год «Козинский Як. Бенц.» арендует завод на Копыловской, 63 у «наследников Булышкиной».\\

\noindent\textbf{А. ДЕНИСОВЪ} (вариант: \textbf{А. ДЕНИСОВ. М. БАРИШ\-ПОЛЬСКІЙ}) – унтер-офицер Андрей Павлович Денисов. Завод в Пирогово существовал в 1896-1913 годах. 1900 год: 5 миллионов кирпичей. В 1903 на заводе работал 81 человек.\\

\noindent\textbf{А. ДОЛОМАКИНЪ} – Адриан (Андрей) Фомич Доломакин, купец. Сыновья Доломакина – Лев и Тарас. Завод в Корчеватом с 1862. На 1894: 1 000 000 кирпичей, оборот 12 000 рублей, 52 рабочих. 1903: 60 рабочих.\\ 

\noindent\textbf{А.Г.}\\

\noindent\textbf{А. ЗАКЪ} – Авраам Ицкович Зак. В 1909-1913 арендовал у Журавлева на Черной горе (Большая Васильковская, усадьбы 140-141) завод, открытый в 1870-м.\\

\noindent\textbf{А.К.РЕЙХЕ} – завод Киево-Софийского Митрополитанскаго дома, в Протасовом яру (усадьба номер 46 на 1894 год). 1900: арендован Августом Рейхе, 41 рабочий из пришлых. 1903: арендатор прежний.

После революции Август Рейхе перебрался в Терийоки (до революции – Россия, после – Финляндия), и в 1930-х отправлял в парижскую газету «Возрождение» письма о жизни русских на его новой родине. Сейчас Терийоки – российский город Зеленогорск.\\

\noindent\textbf{А. КОЗИНСКІЙ} – Абрам Давыдович Козинский, купец 2-й гильдии. Завод основан в 1850 или 1857 году отцом купца, Давыдом. На 1894, адрес «Предместье Куриневка, 22». По другим данным на тот же год – улица Сырецкая, 20 (ясно, что смежная усадьба); выпуск 2 000 000 кирпичей, оборот 32 000 рублей, 72 рабочих. 1900: 80 рабочих (20 местных). 1903: 58 рабочих. На 1911-2 года адрес завода – Сырецкая 2/17, 78 рабочих. 1911: арендатор Сундуковский С. М.\\

\noindent\textbf{А.Л} – А. Лонский либо Аврам Эль Исеевич Лейченко. Лонский был среди совладельцев завода на Демиевке в 1903. У Лейченко работал завод в Василькове. См. главу «Производители».\\

\noindent\textbf{А.С. ЛУНЕВЪ} – купец Афиноген Степанович Лунев. Завод в Мышеловке с 1867 или 1879. Два завода в Пирогово с 1872. 

Завод в Пирогово, 1887: 1 000 000 кирпичей, оборот 18 000 рублей, 45 рабочих. 1894: арендатор «Ив. Игн. Соломотин». 1894, для обоих заводов в Пирогово общее производство кирпича: 1 375 000 штук, оборот 22 000 рублей, 80 рабочих.

Завод в Мышеловке, 1887: 1 000 000 кирпичей, оборот 18 000 рублей, 45 рабочих. Похилевич в книге того же года издания «Уезды Киевский и Радомысльский»\cite{pohyluezd} пишет, что завод Епишкина оценивается в 3 500 рублей. 1894: 800 000 кирпичей, оборот 13 000 рублей, 42 рабочих. 1900: 2 завода (в Мышеловке и Пирогово), 12 миллионов кирпичей.

В 1903, наследники Лунева, Луневы Федор и Кузьма Афиногеновичи, владели двумя заводами – в Мышеловке (96 рабочих) и Пирогово (48 рабочих).\\

\noindent\textbf{А С} – Андрей Слепушов, см. клеймо \textbf{СЛѢПУШОВЪ}.\\ 

\noindent\textbf{Бр. Зарембскіе} – братья Константин Гаврилович и Марцел (Марциан) Гаврилович Зарембские. Завод основан в конце 19 века, на Кирилловской 88 (плоская сторона улицы), с глинищем по Кирилловской, 69, затем 71 (горная сторона улицы). Номер усадьбы в 1898 году сдвинулся с 69 на 71, и по крайней мере между 1907 и 1911 годами номер 69 был за усадьбой усадьбой купчихи Екатерины Соколовской, а 71 – Зарембских. В 1915 году усадьбами номер 88 и 71 уже владел Яков Бернер, когда выкупил у Зарембских кирпичное дело. Про сдвиг нумерации усадеб Соколовских и Зарембских можно судить еще по тому, что в 1882 году усадьба Соколовских имела номер 67, далее в некоем году (не помню) в усадьбе 69 находился кожевенный завод Зарембских. При сдвиге номеров на один, усадьба Соколовских стала 69-й, Зарембских 71-й.

1894: арендаторша «Ан. Ст. Орликова». 1900: 3 миллиона кирпичей. 1911: арендатор Калашников М. И., количество рабочих 90. Позже завод был присовокуплен к предприятиям Якова Бернера.\\

\noindent\textbf{Бр. МОЗГОВЫХЪ} – братья, наследники Иллариона Федоровича Мозгового. У Мозгового в 1903 году был завод в Витачево, на 50 рабочих.\\

\noindent\textbf{Бр. Кирьяновы} – завод возможно в Вышгороде. В начале 20 века держали контору в Гавани. На 1915 год у И. Я. Кирьянова с Чернояровым Л. Б. была общая контора на Набережно-Крещатицкой, 11.\\

\noindent\textbf{Б.\underline{у} (кружок)} – Булышкины. Завод на Куреневке основан в 1852 Тимофеем Булышкиным, в 1890-х и по 1903 включительно, заводом владела Александра Даниловна Булышкина (см. клеймо «А.Б.»). С 1903 наследниками (Александрой Даниловной?) сдан в аренду Якову Бенциановичу Козинскому.\\ 

\noindent\textbf{Б.К.}\\

\noindent\textbf{ВЧ} (курсивом)\\

\noindent\textbf{Б.Ф.} – баронесса Елена Николаевна Фиркс, жена Фиркса Я. В.. Завод на Нижнеюрковской 4 и 6. Вероятно, был расположен по месту бывшего завода Романовского (см. «Производители»). В 2016 год его место можно частично соотнести с военной частью на Нижнеюрковской. 

1894: арендатор  Дмитрий Петрович Павлюченко.

1900: 70 рабочих, из них 20 местные, управляющий крестьянин Матвей Лукин, 4 миллиона кирпичей. 

1902: арендатор тюремное ведомство.

1911: арендатор Герчиков, 60 рабочих. 

Адрес на 1912 прежний. Вероятно, к этому заводу относится и клеймо \textbf{ЮР. КЕРАМИЧЕСКИЙ ЗАВОД Б.Ф.} (трактую как Юрковицкий керамический завод Баронессы Фиркс).

За помощь в создании на Юрковице Макарьевской церкви баронесса была почетным членом Свято-Макарь\-евского братства, куда также входили житель Юрковской улицы художник Николай Пимоненко, инженер кирпичного завода Якубенко А. Ф. (непонятно, завода Фиркс или Рихерта), подольский священник Едлинский и многие другие.\\

\noindent\textbf{В}\\

\noindent\textbf{В. ЕПИШКИНЪ} (вариант: \textbf{В.В. ЕПИШКИНЪ}) – коммерции советник и кавалер Епишкин Василий Васильевич. Завод в Мышеловке был основан в 1873 году. Кирпич оттуда использовался в постройках Печерской крепости. 1887: 900 000 кирпичей, оборот 14 000 рублей, 40 рабочих, стоимость завода 1 200 рублей.

Василий Васильевич Епишкин был похоронен, как и многие из рода Епишкиных, в усыпальнице под церковью во имя иконы Божией Матери «Всех Скорбящих Радость» в деревне Едрово (Валдайский уезд), на Новгородщине. Церковь построена в 1853 году на средства купца Василия Федоровича Епишкина. В усыпальнице покоился прах: потомственного почетного гражданина Валдайского купца 1-й гильдии Епишкина Василия Федоровича, потомственного почетного гражданина Епишкина Василия Васильевича, потомственного почетного гражданина Епишкина Ивана Васильевича, коммерции советника и кавалера Епишкина Василия Васильевича. После революции храм разграбили, помещением пользовались для сторонних нужд. На 2016 от церкви уцелела только коробка без крыши.\\

\noindent\textbf{В. и Б}\\

\noindent\textbf{ВК}\\

\noindent\textbf{ВтБ} – Василий Егорович Терехов-Багреев, купец. Завод основан в 1845 году.

В первой половине 19 века существовал некий завод Терехова с глинищем у Провалья (окрестности Зеленого Театра)\footnote{На плане Киева 1833 года я видел некие «кирпичные заводы» севернее Лавры, под склонами между урочищем Хрещатик (внизу, там где памятник Магдебургскому праву) и Аскольдовой могилой, иначе говоря от моста Метро до Пешеходного моста. Именно там должен был находиться и завод Багреева, и если на плане именно он, то размах производства просто огромный!}. 

На плане 1847 года обозначены «Кирпичные заводы купца Терехова» на Трухановом, на юго-восточном конце того крыла острова, что лежит параллельно Долбычке, на запад от нее. Короче говоря где-то тут: 50°27'02.7"N 30°33'22.8"E. Это точно напротив Провалья, значит глину доставляли на остров по Днепру.

Адрес же завода в 1894 году – Кирилловская, 61. По другим сведениям, завод на Кирилловской 61 в 1894 году принадлежал купчихе Ксении Ивановне Тереховой-Багреевой.

Несмотря на покупку этого завода на Кирилловской Ионой Зайцевым в 1893-м, на 1894 есть данные, что владелец все еще купец Терехов-Багреев Василий Егорович, завод выпустил 500 000 кирпичей, оборот составил \mbox{5 000} рублей, при 18 рабочих.

На 1887: 500 000 кирпичей, оборот 6 000 рублей, 30 рабочих. Завод куплен Ионой Зайцевым в 1893 году для постройки на его основе собственного завода. Там же – знаменитая Кирилловская стоянка, открытая археологом Викентием Хвойкой.\\

\noindent\textbf{Г.ЖУРАВЛЕВЪ} – Георгий Максимович Журавлев и завод Журавлевых (основан Федором Максимовичем Журавлевым в 1870) на Большой Васильковской, 141. Однако известен и Журавлев И. М. при том же заводе в 1912. См. главу «Производители».\\

\noindent\textbf{Г.К.МИНУТЪ} – Генрих Карлович Минут. Наверняка имеет отношение к заводу на Демиевке «Минут и Ко» (упоминается в 1910-1912 годах). Минут как владелец завода на Демиевке проходит и в 1913-м. На 1900 г. «Минут» произвел 3 миллиона кирпичей.\\

\noindent\textbf{Г. С. ЯСЬКО} – Григорий Саввич Ясько. Владел заводами в Мышаловке и Пирогове в 1872-1914 годах. На 1890-й оборот (только Пироговского завода?) – 12 000 рублей. 

На 1894, для завода «при Корчеватом»: 750 000 кирпичей, оборот 11 000 рублей, 40 рабочих. На тот же год, для завода в Мышеловке: 750 000 кирпичей, оборот 9 000 рублей, 40 рабочих.

1900 год – на заводах Ясько произведено 8 миллионов кирпичей.

На заводе в Пирогово в 1903 году трудилось 78 рабочих. В 1911 году указаны заводы в Демиевке и Мышеловке, в них 46 рабочих всего. 1910-1912: контора на улице Деловая, 3. Завод в Мышеловке в советское время стал Корчеватским комбинатом строительных материалов.\\

\noindent\textbf{Г.Я.}\\

\noindent\textbf{ДОЛОМАКИНЪ} – Лев Адрианович Доломакин, арендовал у Якова Бернера один из заводов в Корчеватом. А в 1913 арендовал у Бернера же завод в Мышеловке. Тогда же владел заводом на Демиевке. Адрес оного на 1915: Демиевка, Большая Васильковская, 15.\\

\noindent\textbf{ДОЛОМАКИНА}\\

\noindent\textbf{ДОЛОМАКИН И.З.С.А.Ф.}\\

\noindent\textbf{Д. ХУРГИНЪ и А. ВЕРОЗУБЪ В КІЕВЕ} – Кафельно-гончарный завод Товарищества Давида Лазаревича Хургина и Артемия Борисовича Верозуба на Константиновской, 71 основан в 1879 году. На 1911-й – 80 рабочих. В 1926 году был известен Штейнгутовский Валяно-Глазурный и Гончарный завод на Борисовской, 21 – там трудилась артель «Верозуб, Хургин», производя 15 000 труб в год.\\
 
\noindent\textbf{З.М.Т.} (М совмещена с Т, затем снова следует отдельное Т) – завод Моисея Ив. Тойбы, на 1907 год известен по адресу Юрковская, 43.\\

%\noindent\textbf{И.А. ПОЛОВИНЧИКЪ} – есть такое село Половинчик в Черкасской области, южнее Тетиева, рядом с Монастырищем. А у Монастырища был кирпичный завод, однако он там и остался поныне.\\

\noindent\textbf{И. А. СНѢЖКО} – Илья Адамович Снежко, сын Адама Снежко.

Ставивший клейма, сам был однажды заклеймлён, однако позором. Про Илью Снежко ходила по Киеву слава как о «ловеласе». И вот накануне выборов в городские головы, член управы Илья Снежко был опозорен генерал-губернатором Михаилом Драгомировым, заставившем Снежко прилюдно прочесть вслух жалобу о домогательствах, поступившую от вдовы, искавшей в Киеве работу и обратившуюся за помощью к Снежко. С этим делом связан также писатель Александр Куприн, знакомый с Драгомировым и в ту пору сотрудничавший с газетой «Жизнь и искусство», куда вдова сперва послала письмо с обличением Снежко.\\

\noindent\textbf{ИБ.} – Иван Данилович Байдашников (Бойдачников, Бог\-дашников), мещанин. Завод на Соломенке (Кузьмин Яр, 3 (вероятнее Кучмин, не Кузьмин); вариант: предместье Верхнее Соломенное, дом №3) известен в 1887-1900 годах. 1894: 400 000 кирпичей, оборот 4 000 рублей, 13 рабочих. 1900: 14 пришлых рабочих.

Известен также завод владельца {Бойдачникова Ивана Даниловича} в предместье Верхнее Соломенное №147, учрежденный в 1875 году. 1887: 250 000 кирпичей, оборот 3 000 рублей, 12 рабочих.

Не исключено, что это один завод, однако его усадьбе в разное время соответствовали разные адреса.

На плане 1896 года завод Байдашникова обозначен однако на юго-восток от Галаганов, по другую сторону железной дороги от них на некотором расстоянии.

Вероятно тот же владелец (Богдашников Иван Данилович) на 1903 год у завода в деревне Совках, на 11 рабочих.\\

\noindent\textbf{И. В. КНОТТЕ} – Иван Вильгельмович Кнотте. Завод в селе Халепье Киевского уезда.

Братья Иван и Влацлав Кнотте – поляки, помещики села Сидавы Винницкого уезда Подольской губернии, где владели 167 десятинами земли. На 1911 год жили в Киеве по адресу Никольско-Ботаническая, 31.\\

\noindent\textbf{Ил. МОЗГОВОЙ} – Илларион Федорович Мозговой. Завод основан в 1898 году в селе Витачев Стайковской волости Киевского уезда, на берегу Днепра, известен по 1915. 1903: 50 рабочих. На 1909 контора в Гавани.\\ 

\noindent\textbf{И. КАЛАШНИКОВЪ} – Калашников Иван Максимович. Завод известен в 1906 году. См. главу «Производители» про «Калашникова Митр. Ивановича».\\

\noindent\textbf{М. Я. БЯЛИКЪ} – вероятно связан с Лейбой Бяликом (см. клеймо \textbf{Л.Б.})\\

\noindent\textbf{І.М.З} – Иона Мордкович Зайцев (1828-1911), почетный гражданин. Завод на Кирилловской, 61. На 1900, 98 рабочих (25 местных), управляющий мещанин Логвин Бондарь. На 1911 – 103 рабочих. 

Местность известна по Кирилловской стоянке, найденной в глинище завода, и по делу Бейлиса, где обвиняемым в ритуальном убийстве мальчика Андрея Ющинского был служащий завода Менахем Мендель Бейлис. 

Доходы с завода шли на еврейскую больницу, расположенную между глинищем и Кирилловской улицей (ныне по  адресу Кирилловская 61 и 63). 

В 1913 году владельцем завода выступает  «Хирургическая лечебница учрежденная И. М. Зайцевым», указаны два адреса: Кирилловская 61 и Верхне-Юрковская 32 (усадьба выходила на верх горы, от Кирилловской до Верхне-Юрковской); управляющий Дубовик Х. Б., производство 5 000 000 кирпичей, 150 рабочих.

На месте глинища на 2017 год расположена автобаза по адресу Богуславский спуск, 1.\\

\noindent\textbf{КА}\\

\noindent\textbf{К:БУ:} (вариант: \textbf{К:БУ})\\

\noindent\textbf{КИРНИЦКІЙ} – мещанин Прокофий Дмитриевич Кирницкий. Завод в Халепье Стайковской волости Киевского уезда.\\

\noindent\textbf{К.З. МЫШАЛОВКА} – Кирпичный Завод Мышаловка.\\

\noindent\textbf{К. \& Ш.}\\ 

%\noindent\textbf{КПЛ 1848} – завод Киево-Печерской лавры. См. главу «Производители».\\

\noindent\textbf{К.С.Г.І.}\\

\noindent\textbf{Л.Б.} – Людмила Павловна Баранова. Дымерский завод, 1895-1899 годы.\\

\noindent\textbf{Л.Б.} – Лейба Яковлевич (Янкель) Бялик, действительный член Юго-Западного отделения Российской Экспортной Палаты (1913). Завод на Демиевке в 1910-х.\\

\noindent\textbf{Л. Д. ГЕРЧИКОВЪ} (вариант: \textbf{Л. ГЕРЧИКОВЪ}) – Герчиков Лейба Дувидович, купец 2-й гильдии. В 1911 Герчиков арендовал завод у баронессы Фиркс, там трудилось 60 рабочих – вероятно на то же время и хотя бы по 1914 год справедливо и клеймо.\\

\noindent\textbf{Л. ДОЛОМАКИНЪ} – Лев Адрианович Доломакин, арендовал у Якова Бернера один из заводов в Корчеватом либо Мышеловке. Однако за 1912 год сведения таковы – завод в Демиевке, с конторой на Большой Васильковской, 15.\\

%\noindent\textbf{Л.К.З.} – завод Киево-Печерской лавры. См. главу «Производители».\\

\noindent\textbf{Л.П.Ч.} (вариант: \textbf{Л.Ч.}) – Лазарь Павлович Чернояров, купец 1-й гильдии. Было пять кирпичных заводов, основанных семьей Чернояровых в 1885-90 годах – три около Межигорья и Вышгорода, два в 1896-97 в Вытачеве и Халепье (Холопье). 

На 1900 год у Чернояровых, возможно, осталось только 4 завода, ибо есть сведения, что на четырех заводах в тот год было произведено 15 миллионов штук кирпича.

В 1903 году на «вышгородских» заводах трудилось 315 рабочих и стояла паровая машина в 6 сил, в Вытачево работали 57 человек, в Мышеловке 45.

В 1913 году заводом в Халепье владел Чернояров В. Л., управлял Ляхтин М. А., стояла паровая машина в 46 сил, производилось 3 000 000 кирпичей, при 90 рабочих.

В списке «Личный состав служащих Киевского второго женского училища за 1899-1900 год» указано: «Почетный блюститель по хозяйственной части 1 гильдии купец Лазарь Павлович Чернояров.». На стыке 19 и 20 веков, Лазарь Чернояров купно с Яковом Бернером состояли в Киевском Свято-Владимирском братстве ревнителей православия и были членами его Совета.\\

\noindent\textbf{Л.Д.}\\

\noindent\textbf{Л. Я. БЯЛИКЪ} – Лейба Яковлевич Бялик. Завод на Демиевке, с 1910 года.\\

\noindent\textbf{М} – Михельсон Фридрих Густавович (1840-1908), купец 1-й гильдии. Завод по улице Сырецкой, 25 известен в 1897-1912. На 1911 я встречал адрес Сырецкая, 66.

В 1899 году настоятельница Покровского женского монастыря Каллисфения во время сооружения соборного храма писала купцу:

\begin{quotation}
Милостивый государь Фридрих Густавович! Третий год уже на исходе, как Вы поставляете кирпич для постройки храма в Киево-Покров\-ском женском общежительном монастыре. 

Несмотря на то, что цены в течение этого времени в г. Киеве на материал растут с каждым днем, Вы, будучи душевно расположены к нашему благому и святому делу, поставляете свой кирпич по той же цене, по какой поставляли и в начале постройки храма.
\end{quotation}

1900: 200 рабочих (50 местных, 150 пришлых), произведено 7 миллионов кирпичей. 1903: 200 рабочих. 

На 1911 завод уже во владении Наследников Михельсона, адрес Сырецкая 25/34, арендатор «Сундуковский Э.М.», 129 рабочих. На 1912 адрес прежний. 1913: адрес прежний, арендатор Сандуковский. На 1915 адрес прежний, владельцы «наследники Михельсона».\\

\noindent\textbf{М} (изящная) – Графиня Морочинская Мария Ивановна. Завод в Тетиеве.

Морочинская – землевладелица, в Ушицком уезде ей принадлежало 1 218 десятин в селе Жабинцы и местечке Замехов.\\

\noindent\textbf{Мас}\\

\noindent\textbf{МВ} (в ромбе)\\

\noindent\textbf{МБ} – Моисей Мордкович Бройда (Бройде), купец. В 1870 году Бройда арендовал у купца Валуева в Вышгороде завод, по крайней мере по 1887 год. 1887: 900 000 кирпичей, оборот 12 000 рублей, 30 рабочих. В 1894 Бройде арендовал завод Управления Государственным Имуществом около Вышгорода.\\ 

\noindent\textbf{МЕЖИГОРЬЕ Е.Ш.} – быть может, к этому имеет отношение Е. Шуминская (см. главу «Производители»).\\

\noindent\textbf{М.иМ.}\\

\noindent\textbf{М.Н.} – купец Макар Иванович Николаев. Завод в Новых Петровцах, известен с 1878. 1887: 326 000 кирпичей, оборот 5 000 рублей, 14 рабочих.\\

\noindent\textbf{М. РИХЕРТЪ} – Михаил Вильгельмович (Васильевич) Рихерт, купец 2-й гильдии. Адрес завода на 1911-2 – Юрковская, 2.
Современный адрес: Нижнеюрковская, 2.

На углу тогдашнего, не срытого еще склона Лысой горы (современной Юрковицы) и по оврагу вдоль него, где теперь улица Нижнеюрковская, в 18 веке работали три завода – Кирилловского монастыря, мещанина Ивана Григоровича-Барского и Гудима, причем Григорович и Гудима арендовали у монастыря землю. 

Дело этих заводов запутанное, ибо по одним данным, Гудимы приобрели у монастыря завод, по документам же монастырский завод и завод Гудимы существовали одновременно, при этом мне неясно, как подружить с ними на местности завод Григоровича. За дополнительными сведениями, или путаницей, отсылаю вас ко главе «Производители», однако точно можно считать, что на куске участка, где потом работал завод Рихерта, с 1770-х годов существовал завод Гудимы, впоследствии – влиятельной дворянской семьи Гудимов-Левковичей (см. главу «Производители»). 

На карте 1886 года еще виден «кафельный и кирпичный завод Гудимы». Он стоял ближе к «рогатке» нынешнего перекрестка улицы Нижнеюрковской с переулком Нижнеюрковским, а на углу Кирилловской и Нижнеюрковской был пивзавод Псиола.

На 1887 год кирпичным заводом владела дворянка Екатерина Матвеевна Гудим-Левкович, завод по адресу Кирилловская улица, 2, произвел 230 000 кирпичей, годовой оборот имел 5 000 рублей, при 9 рабочих. На 1890 завод этот, выпускавший «строевой и огнеупорный кирпич», «на Юрковице», имел годовой оборот 6 000 рублей.

В 1894 году Михаил Вильгельмович Рихерт приобрел завод Псиола и взял кирпичный завод у Екатерины Гудим-Левкович в аренду (по другим источникам, купил). На 1894, владелицей указана Екатерина Гудим-Левкович, Рихерт записан заведующим и арендатором, адрес Нижне-Юрковская, 2, производство 300 000 кирпичей в год, оборот 5 000 рублей, 16 рабочих.

В 1895-м при расширении производства построена каменная печь Гофмана\footnote{Печь Фридриха Эдуарда Гофмана, для обжига кирпича, с высокой дымовой трубой. Первую печь Гофман запатентовал в 1858 году. Обустройство такой печи на заводе обходилось недешево, но и окупаемость была нехилой – кроме больших партий загрузки кирпича-сырца в печь, гофмановская печь выгодно отличалась от «стенных» печей меньшей прожорливостью топлива. Для обжига 1 тысячи кирпичей-сырцов в обычной стенной печи требовалось 20 пудов каменного угля, в гофмановской – 12. В гофмановской печи было не менее восьми отделений (в цикле работы, три отделения-камеры подогреваются, три остывают, два разгружаются), иногда по 16. В каждое отделение помещали не менее 1500 сырцов. При виде сверху гофмановская печь похожа на апельсин, где дольки это отделения для сырца, а посередке торчит труба. Топливо – раскрошенный каменный уголь или дрова –  сыплют сверху непосредственно на уложенный сырец.}. Такая печь была и на заводе Зайцева.

1900: 70 местных рабочих. 1903: 80 рабочих. 1911: 58 рабочих. После гражданской войны завод обанкротился. Его восстановили пленные немцы и венгры. 

В 1915 году указан адрес «Кирилловская, 35» – это усадьба смежная с Нижнеюрковской, 2.

На основе завода создано предприятие «Керамблоки», в 1932 – перепрофилирование на выпуск канализационных труб. В 1948 году – выпуск облицовочной плитки. После Великой Отечественной войны в усадьбе завода были общежития для рабочих, клуб с любительским театром, патефоном и, позже, телевизором. 

К 1950-м доля производства кирпича увеличилась до 85 процентов. Продолжался выпуск плитки разных видов, в том числе мозаичной глазированной. В 1965 году завод начал пожирать остатки юго-восточной части Лысой горы, ныне известной как Юрковица\footnote{В Киеве несколько Лысых гор – одна на Зверинце, другая на Воскресенке, третья – на Кирилловских высотах. Название Юрковицы переползло на нее с соседнего холма, там где кладбище старообрядцев.}.

1978 год – кирпичные блоки для детских садов. 1980-е – спад производства. На 2016 год завод не работает по назначению, заводские помещения сдаются в аренду, последнее глинище заполнено водой – там замусоренное озеро.

По крайней мере с 1879 года известен также кирпичный завод Рихерта на Сырецкой 23, 24. Отец братьев Рихертов – Михаила и Якова – Вильгельм, владел на Куренёвке, по современному адресу Сырецкая, 19, мощнейшим пивоваренным заводом (теперь это Киевский завод шампанских вин «Столичный») и смежным с ним кирпичным. После смерти Рихерта-старшего, пивзаводом владели оба брата, а потом Михаил начал развивать заводы на Нижнеюрковской, а сырецкий пивзавод достался Якову. Глинище кирпичного завода на Сырце – поныне на территории завода шампанских вин. Чьим считался кирпичный завод на Сырце после смерти старшего Рихерта, я не знаю.\\

\noindent\textbf{М. ФОКИНА} – Фокина Мария Кирилловна. В 1906 году арендовала завод Александра Адамовича Снежко. До этого у него же, в конце 19 века, арендовал завод Илларион Алексеевич Фокин.\\

\noindent\textbf{М. ХЕЛЕМСКІЙ} (вариант: \textbf{М.Х.}) – Хелемский Мойша-Рувим Пинсахович. Завод в Новых Петровцах основан в 1876. На 1900: 3 миллиона кирпичей.

На 1903 году в Новых Петровцах известен завод Хелемской Суры Вольфовны с 80 рабочими.\\

\noindent\textbf{М. Я. ГРЕБЕНЪ} – завод известен в 1899-1909 годах. На 1907-й его адрес совпадает с заводом братьев Зарембских, по Кирилловской 88, при единовременном сосуществовании и завода братьев.\\

\noindent\textbf{Н. БЫСТРИЦКІЙ} – «Быстрицкий Нух. Лейз.». Завод с 1912 года.\\

\noindent\textbf{Н З} – Николай Зелинский? Завод Зелинского известен в 1872.\\

\noindent\textbf{Н. СНѢЖКО} – Наследники Снежко. Наследники Адама Андреевича Снежко: Флорентий, Александр, Андрей, Илья Снежко. Заводы купца Адама Снежко (умер 28 октября 1874) известны с 1862 года возле Новых Петровцев и Корчеватого. На 1887 год, завод в Корчеватом выпустил 1 000 000 кирпичей, имел оборот 18 000 рублей и 42 рабочих. На 1900 год, 4 завода наследников Снежко в Китаеве произвели 15 миллионов кирпичей.

В 1903 году на заводе Снежко Ильи Андреевича и Снежко Флорентия Александровича в Новых Петровцах работало 60 человек.\\

\noindent\textbf{П.и.Г.} (курсивом) – возможно, какой-то васильковский завод.\\

\noindent\textbf{П.О.} – Петровицкое Общество (крестьянское). Межигорье, Новые Петровцы, завод существовал в 1874-1917. В 1894 и 1909 годах арендаторами указаны Семен Тимофеевич Горенко и купец Мошка Пейсахович Хелемский (М.Х.).\\

\noindent\textbf{ПОПИРНЯ} – завод Гольдфарба Константина Константиновича (купец 2-й гильдии, христианин) в Горенке, известен в 1906. Папирня это другое название Горенки (речки и селения), от стоявшей там бумажной фабрики Братского монастыря.

В 1887 году в Папирне был известен большой кирпичный завод купца Петра Астахова, купившего эту деревню у наследников киевского войта Михаила Григоренко за 18 000 рублей. На 1887 купец уже впрочем числился покойным, однако завод тогда существовал. Вероятно, между этим заводом и заводом Гольдфарба связь самая прямая, а клеймо могло быть заведено еще при Астахове.

В 1913 году Гольдфарб владел также заводом в селе Мостище. У него был также дом в Киеве, на Мало-Васильковской улице\\

\noindent\textbf{П.З}\\

\noindent\textbf{П.К.}\\

\noindent\textbf{С. В. ШАТОВА} – Сергей Васильевич Шатов, владелец завода (к юго-востоку от завода Субботиной и нынешнего озера Глинки) Юлии Шатовой в 1901-1913 годах. 1900: 50 пришлых рабочих. На 1911 есть данные: арендатор Зак. А. И., адрес указан Б.Васильковская 141, однако это адрес завода Журавлева, у Шатовых был номер 151.\\

\noindent\textbf{С.Г.} – Семен Тимофеевич Горенко, купец, на 1912 входил в состав Киевского Отдела Императорского Православного Палестинского Общества.

Завод в Новых Петровцах. 1894: 80 рабочих.\\

\noindent\textbf{С. КОБЛОВЪ} – Коблов Сергей Яковлевич – завод в Стайках. Другие данные – завод при селе Волки с 1830, для него в 1894 указано производство 750 000 кирпичей, оборот 9 000 рублей, 25 рабочих.\\

\noindent\textbf{СЛѢПУШОВЪ} – завод купца, старообрядца Андрея Евдокимовича Слепушова (1842-1915) в Василькове. Построен в 1883 в саду усадьбы купца. Ныне это окрестности по адресу ул. Тимирязева, 61 (дореволюционный адрес: Георгиевская, 11). 

Завод Слепушова производил кирпич обычный и огнеупорный (без клейма). Другое клеймо завода: \textbf{А С}. Кирпичи Слепушова из Василькова находят и в Киеве.

Слепушов перебрался в Васильков из Киева в 1883 году (а в Киев прибыл из Рязани), но еще какое-то время жил на Подоле в доме матери своей жены.

Кусок выписи из Крепостной Киевского нотариального архива книги по городу Василькову, Киевской Губернии, 1883 год, №17, часть 1, страница 20 №10: 

\begin{quotation}
Тысяча восемьсот восемьдесят третьего года, марта пятнадцатого, явились к Владимиру Никифорову Юрьеву, Васильковскому нотариусу, в контору его, находящуюся в Здоровецкой части, в доме Левандовской, лично ему известные имеющие исконную правоспособность к совершению актов, мещане: Яков Никифоров Ляхин, действующий от имени коллежского асессора Семёна Ефремова Грабовского, на основании доверенности, явленной у него, нотариуса, 4 апреля 1882 года, за №190, представленной ему в подлиннике, и Андрей Евдокимов Слепушов, живущие: 1-й в гор. Василькове, а 2-й в гор. Киеве, в собственных домах, в сопровождении лично ему известных свидетелей: коллежского регистратора Ефима Яковлевича Лебеденко, мещан Берка Юсифова Пиковского и Ильи Никитова Мещерака, живущих в г. Василькове: 1-й в доме Шурупова, 2-й в доме Карагинского и 3-й в своём доме, с объявления, что они, Ляхин и Слепушов, заключают договор о продаже недвижимого имения на следующих условиях: Яков Никифоров Ляхин, продаёт мещанину Андрею Евдокимову Слепушову за четыреста пятьдесят рублей принадлежащую доверенностью Ляхина, коллежскому Ассесору Семёну Ефремову Грабовскому, леваду, состоящую в предместье г. Василькова в «Здоровки», обведённую вокруг канавою, заключающую в себе земли около шести десятин, граничащую:с востока – почтовой дорогой, ведущей из г. Василькова в гор. Киев и землею солдата Бирюкова, с запада – проулком, пролегающим из г. Василькова к мельнице Шварцевой, с севера – хутором мещанина Гаповаленкова, с юга – огородами мещан Тенедкинского и Слюсаренко, доставшаяся ему, Грабовскому, покупкою с публичного торга в Васильковском <неразборчиво> Мировым Судьёй, по данной крепости, совершённой Васильковским Нотариусом Юрьевым 21-го апреля 1882 года в №35, отмеченной 8 мая того же года в реестре крепостных дел Киевского Нотариального Архива, часть 1, страница 51. [...]
\end{quotation}

На форуме «Тайны города Василькова» я также встретил несколько других предположений о месте завода. Первое, что он был по месту нынешнего Васильковского профессионально-технического лицея, у яров за Декабристов, 43. Второе – на улице Ворошилова, около усадеб 16, 14 – однако мнение это оказалось ошибочным, там жил мещанин Слюсаренко, увлекавшийся гончарством и владевший печью для обжига.\\

\noindent\textbf{С. М. ОСМОЛОВСКІЙ} – Семен Михайлович Осмоловский. В 1894 году арендовал в Мышеловке завод, принадлежащий «Госуд. Имущ.». По некоторым сведениям, позже с этим клеймом выпускались кирпичи на Кирилловской улице.\\

\noindent\textbf{СНѢЖКО ХЛЕБНИКОВА} – Хлебников был управляющим на заводах Снежко, потом породнился с семьей, женившись. В девичестве была Елена Адамовна Снежко, а стала Хлебникова. На бывшем доходном доме семьи Снежко и Хлебниковой (угол Пушкинской и площади Льва Толстого) буквы «Х», вписанной в «С». Тут же на площади стояли бани Михельсона, но здание до наших дней не дожило.\\ 

\noindent\textbf{С.Ш.} в рамке.\\

\noindent\textbf{C.Ш.} без рамки – такие кирпичи находят в Пуще-Водице.\\

\noindent\textbf{С. ШЕВЧЕНКО} – «Семен Фед. Шевченко». 1913: адрес Кирилловская, 47. 1914: адрес Верхне-Юрковская, 14; производство: 750 000 кирпичей, 100 рабочих.\\

\noindent\textbf{Тюр.Кирп.Зав.} – Тюремный Кирпичный Завод.\\

\noindent\textbf{Х. ВОЛКОВЪ} (вариант \textbf{Х.Ш. ВОЛКОВЪ}) – заводы купца 1-й гильдии Хаима Шмуля (Тевелевича) Волкова, на 1911 год «при  с. Корчевате, Межигорье и в Киеве». Волков с 1906 года брал в аренду заводы С. В. Шатова и Ф. А. Снежко. Волков торговал строевым кирпичом и железняком\footnote{Пережженный кирпич, которым мостили улицы и выкладывали над крышей трубы. Остеклованный по краям, местами с темными пятнами.}, в разное время держал контору на Владимирской, 41, на Киево-Воронежском шоссе, 3 (в 1909), на Фундуклеевской 26 (в 1911). 1909: 12 000 000 кирпичей (статистика по двум заводам в Китаево?). 1913: указан адрес Кирилловская 71/88, а это братья Зарембские. 

В 1913 году Волков арендовал у Якова Бернера завод в Мышеловке, имел заводы в Межигорье, Китаево, Корчеватом.\\

\noindent\textbf{Х. МАРРЪ} – завод (производил «белый строевой кирпич») купчихи 3-й гильдии Христины Семеновна Марр (1827-1888), стоял на Сырецкой, 22 с 1861 года. 

На 1882 указан адрес Копыловская, 30\footnote{Улица Копыловская пересекает улицу Сырецкую.}. Рядом – Генрих Марр, Копыловская 30; Христины Марр также тогда же – усадьбы Кирилловская 22, Сырецкая 22. 

На 1887: 1 600 000 кирпичей, оборот 23 000 рублей, 87 рабочих. 1890: оборот 21 000 рублей.

Христина Марр – немка, перебралась в Киев с мужем-пивоваром Иоханном в 1843 году, земельный участок на Сырце изначально был куплен на ее имя. Там устроили пивной завод. В 1855 году Иоханн Марр умер в возрасте 44 лет, пивоварня перешла к 28-летней вдове, а гостиница «Большая национальная» на углу Крещатика (где был кинотеатр «Орбита») и 10 000 рублей завещались детям, но известно, что в 1870-е именно у Кристины Марр гостиницу берут в аренду.\\

\noindent\textbf{Ф. И. ПОЛЛАКЪ} – купец 2-й гильдии Федор (Франц) Иванович Поллак. 

Владел также банями Троицкими и Галицкими, колбасной фабрикой, пивзаводом «Чехия» на Демиевке, служил директором-распорядителем Демиевского чугунолитейного и механического завода.

Чех, поддерживал Киевскую церковь евангельских христиан баптистов, отдал под богослужения часть своего дома (Жилянская и Караваевская 104/27, где собрания баптистов проходили с 1904 по 1949 годы. Поллак купил деревянный дом и для житомирских баптистов (по ул. Иларионовской (Котовского), 10, дом стоит во дворе школы №3). Ближе к революции жизнь этого довольно крепкого купца, владевшего доходным домом и колбасными лавками, пошла под откос.

\begin{quotation}
Донос о деятельности в Киеве баптистского проповедника Ф. Поллока, В том числе о создании им баптистских учебных заведений. 1916 г.

Его Высокопревосходительству Господину Киевскому Губернатору Александра Федоровича Рипса Заявление

Настоящим имею честь заявить Вашему Высокопревосходительству, что проживающий в г.Киеве по Кузнечной улице в доме № 45 Федор Иванович Поллок в продолжении 30 лет состоит главным руководителем и кассиром существующей секты Баптистов находящейся в его же доме по Жилянской улице № 104. Для распространения этого учения в России Поллок получал из Германии крупные денежные суммы и различные брошуры, которыми снабжал членов общины. Деньги и брошуры получались через Австрийское Консульство Секретарем Консула состоял его зять Р. Н. Динтер. Цель существования и распространения учения Баптистов – в противовес Православию. Пропаганда ведется в широких размерах по деревням, между крестьянами для сближения с колонистами немецкими баптистами с которыми и устраиваются общие собрания и так называемые братанья... Поллок – бывший австрийский подданный по национальности чех, принял Русское подданство и перешел в православие для того, чтобы свободно устраивать свои дела, приобретать имущество в городе и уездах.

На Демиевке по Совской лице Поллок за общественные деньги построил молитвенное собрание и школу баптистов, затем купил имение возле деревень Мироцкое, Микуличи и станция Немешаево I.К.К.Ж.Д. (4000 дес.). Где имел строить Баптистическую школу для подготовки проповедников...

Прошу проверить Ваше Высокопревосходительство мое заявление о вредном направлении Поллока за все время его проживания в России, начиная с 1900 года...

[подпись]

9 февраля 1916 года
\end{quotation}

А после революции, уже после 1919 года Поллак был арестован в Киеве и в качестве заложника находился в Москве в Андрониковском монастыре. Я не смог проследить позднейшую его судьбу.\\

\noindent\textbf{Ф.Б.} – Федор Березовский, завод в Вышгороде.\\

\noindent\textbf{Ф. ВОЛКОВЪ М. ЦИПЕНЮКЪ}\\

\noindent\textbf{Ф.и К. ЛУ (БР) НЕВЫ} – братья Федор и Кузьма Луневы, потомки А. С. Лунева.\\

\noindent\textbf{ФЛ}\\

\noindent\textbf{Ф.Ч дата.} – Чаманская Фридерина (Фредерика) Артуровна, купчиха – завод с 1815 на Кирилловской улице, 49. 1887: 300 000 кирпичей, оборот 2 000 рублей, 15 рабочих.\\

\noindent\textbf{ЧЕРНОЯРОВЫХЪ} – потомки купца 2-й, затем 1-й гильдии Лазаря Павловича Черноярова. Возможно, это клеймо ставил и Чернояров Василий Лазаревич\footnote{Родился в 1857 году, в молодости принимал участие в революционной деятельности, подвергался арестам и ссылкам, к концу 19 века с подачи отца стал заниматься подрядами на железные дороги, в 1898 году вернулся в Киев.}, владевший на 1903 год заводом в Холецах Стайковской волости (148 рабочих, паровая машина в 40 сил). Чернояров В. Л. записан также владельцем завода в Халепье на 1915 год.\\

\noindent\textbf{Ш}\\

\noindent\textbf{ШАТОВА.} – завод Юлии Алексеевны Шатовой (вдовы полковника Шатова), известен с 1846 или 1870. Большая Васильковская 151 (считалось Демиевкой), глинище за глинищем Субботиной в сторону станции Киев II, кирпич красный строевой и желтый, годовой оборот на 1890-4: 20 000 рублей. 1894: арендатор купец «Федор Ник. Поддубный»\footnote{В 1900 некий завод Поддубного выдал 3 миллиона кирпичей.}; производство 1 500 000 кирпичей, оборот \mbox{21 000} рублей, 41 рабочий.

В «Путеводителе по городу Киеву» за 1911 год адресация несколько иная по названию – «Шоссе Киевско-Курской ж.д. [...] Левая сторона: 1. Субботиной Э., 3. Шатовой Ю., 5. Субботиной Э.». В те годы завод получает адрес усадьбы №3 по шоссе то Киево-Курской, то Киево-Московской железной дороги.

1887: 1 000 000 кирпичей, оборот 18 000 рублей, 50 рабочих. На 1890 и 94 годы оборот по 20 000 рублей (в Путеводителе Бублика). 

На 2016 на глинище завода стоит автохозяйство ГУ МВД Украины в г. Киеве, по адресу Железнодорожное шоссе, 9, под склоном Черной горы с семидесятыми номерами частного сектора улицы Менделеева.\\

\noindent\textbf{Э. ШПРУНГЪ} – Эмиль Я. Шпрунг, купец 2-й гильдии. Заводы известны с 1914.

Шпрунг держал контору на Крещатике, к 1915 году торговал чем угодно: металлическими трубами, канатами, кислотой, лампами, сталью, инструментами вроде лопат и напильников, мешками разных видов и прочим. Каталоги рассылал бесплатно.\\

\noindent\textbf{Э и В}\\

\noindent\textbf{Э и Л} – завод Андрея Эрлиха, рижского гражданина, на хуторе Корчеватом Выдубецкого монастыря. Завод был построен в 1872 году для удовлетворения нужд строящегося рядом Лысогорского форта (1871-1877 годы).

По крайней мере одна потерна Лысогорского форта сложена целиком из этого кирпича (длинные, с клеймом на ребре). В сооружениях форта использовались также кирпичи Волкова, Епишкина и других производителей. На Лысой горе мною обнаружены кирпичи с клеймами \textbf{А.К.РЕЙХЕ} и в прямоугольнике «1 что-то там дальше». Там же люди находили кирпичи \textbf{А.С. ЛУНЕВЪ}, \textbf{Я БЕРНЕР}.\\

\noindent\textbf{ЭС, Э.С.} – Эмилия Густавовна Субботина. Кирпичный завод Эмилии Субботиной был на юго-восток от нынешней Лыбедской площади. Озеро Глинка частично занимает место карьера этого завода. На аэрофотоснимке 1943 года видно, что глинищем в горе съеден эдакий круг с островом посередине, и современная Глинка – восточная половина и «рва», и острова. На том же снимке, кирпичный завод лежит от глинища на юг, между глинищем и железной дорогой.

%Сим предприятием с 1833 года владела семья Эйсманов, поначалу аптекарь Иван Федорович Эйсман (Иоганн-Сигизмунд Эйсман, 1794-1862), потом сын его – профессор Университета св. Владимира, миллионер, политик, городской голова Густав Иванович Эйсман, да жена Эйсмана-старшего, Елена-Эмилия.

Сим предприятием с 1833 года владела семья Эйсманов, поначалу аптекарь Иван Федорович Эйсман (Иоганн-Сигизмунд Эйсман, 1794-1862), потом сын его – профессор Университета св. Владимира, миллионер, политик, городской голова Густав Иванович Эйсман, а потом дочь Густава, Эмилия, в замужестве Субботина.

Из кирпичей эйсмановского завода выстроены Университет, здания присутственных мест, Первая и Вторая гимназии, кадетский корпус, Александровский костёл. В 1870-х завод перешел к дочери Эйсмана, Эмилии Густавовне Субботиной. Супруг ея, доктор медицины Виктор Андреевич Субботин (1844-1898), был деканом медицинского факультета и профессором Университета св. Владимира. 

Про завод Эйсмана в «Статистическом описании Киевской губернии»\cite{fundstat} за 1848 сказано:

\begin{quotation}
В Киеве самый значительный из частных кирпичных заводов аптекаря Эйсмана на собственном его хуторе, на выезде большой Васильковской дороги. Он устроен в 1833 году; на первоначальное обнажение кирпичной глины было употреблено в первые годы до 8 000 рублей серебром и на постройку печей и сараев 6 000 рублей серебром. 

После первоначального устройства улучшений никаких введено не было. На заводе ныне находятся 4 печи для выделки кирпича и 7 сараев для приготовления сырца. Последний выделывается в формах, склеенных их четырех дощечек; эти формы делаются мелкими ремесленниками по заказу. 

На ремонт строений ежегодной употребляется по 400 рублей серебром, на ремонт печей от 400 до 600, ремонт глинища от 2 000 до \mbox{3 000} рублей серебром. Завод не застрахован; городских и других пошлин, сборов и повинностей платится до 60 рублей серебром.

Со времени основания завода заготовлялась к работе глина в разных количествах, ежегодно от 300 до 600  700 кубических саженей она вынимается из горы при том же заводе находящейся. Для выпалки кирпича выходит ежегодно от 200 до 1 000 саженей дров, покупаемых от 4 до 6,5 рублей серебром за сажень, без доставки на завод. На тысячу кирпичей употребляется обыкновенно по 1/2 сажени глины и по 1/2 сажени дров.

Выделка сырого кирпича продолжается от 15 Апреля до 15 Сентября, а выпалка целый год. В последние годы с 1844-1847 средним числом выделывалось по два миллиона обозженнаго кирпича, а в 1848 – 1 500 000. Всего же выделано было со времени основания завода более 12 000 000, но несколько лет вовсе работы не производились.

Кирпич выходит, по свойству глины, доброкачественный для кладки стен, фундаментов и труб, ибо не боится сырости; но в печи не годится. Он продается частным лицам и для казенных построек ценою до 10 рублей серебром за тысячу, с доставкою.

На выкопку и доставку глины к месту производства употребляется ежегодно от 600 до \mbox{3 000} рублей серебром, на содержание рабочих от 3 000 до 6 000, а на содержание лошадей и прочего в прежние годы от 1 000 до 2 000, а ныне, в 1848 году, по случаю неурожая, на 40 лошадей вышло 1 500 четвертей овса, на \mbox{3 750} рублей серебром и сена на 1 000 рублей серебром.

Рабочих первоначально было 15 человек, а теперь 150. Работники, выделывающие кирпич сырец, получают за 1 000 по 1 рублю серебром, а прочие помесячно по 2, 3, 4 рубля серебром и продовольствуются пищею от хозяина; первые работают в день от 8 до 10 часов, а последние, находясь при возке глины, воды и т.д., работают по 9 и 10 часов в день. 

Рабочие по большой части нанимаются из казенных крестьян Киевского и Васильковского уездом и из Васильковских мещан; поденьщики нанимаются преимущественно в зимнее время для выкопки глины, по 20-30 копеек серебром в день.

Приблизительный баланс завода можно вывести следующим образом:

\textit{Расход}: на выкопку и доставку глины к месту работ, средним числом – 1 800 рублей серебром.

дрова для выпалки кирпича – 5 250

плату и жалованье рабочим – 4 500 

содержание лошадей – 2 000

ремонт строения, печей и глинища – 2 900

пошлины и повинности – 60

4\% от вкладочного капитала – 560

Всего оборотного капитала: 17 000. 

\textit{Приход}: за два миллиона кирпича, по 10 рублей тысяча – 20 000.

Чистой прибыли: 2 930. 

Что составляет 17\% с оборотного капитала.
\end{quotation}

Хутор Эйсмана в то время занимал местность нынешней Лыбедской площади и ее окрестностей, вместе с мысом холма, у подножия которого теперь озеро Глинка – в первой половине 19 века там еще не была срыта часть холма, а на горе стояли кирпичные сараи.

В 1860-1875 годы завод Эйсмана производил не более 2 000 000 кирпичей в год.

В 19 веке около глинищ заводов Субботиной и Шатовой, профессор Киевского университета святого Владимира Афанасий Семенович Рогович находил куски янтаря, доисторических дубов и хвойных (в частности секвойю), а также остатки древних морских растений. Изучением этих мест занимался также И. Шмальгнаузен\footnote{См. Записки Киевского общества естествоиспытателей, тома 4 и 7.}. Следы наземных растений, обнаруженные в здешней синей глине, относятся ко времени, когда на Киевщине растительность была как нынче в тропическом поясе. Пальмы, смоковницы, секвойи.

На 1879 год к юго-восточной стороне завода примыкал другой кирпичный завод – Шатовой, на месте его карьера теперь автохозяйство ГУ МВД Украины. В «Прибавлениях» к «Путеводителю по Киеву» 1882 года последовательно на левой стороне Большой Васильковской указаны: «Субботинъ (кир. зав), Шатовой (кир. зав.)».

На 1887 год, производство 2 000 000 кирпичей, оборот 30 000 рублей, 60 рабочих. На 1890 годовой оборот завода «Субботина» по адресу Большая Васильковская, 149 – 34 000 рублей, «кирпич желтый строевой и алый». На 1894 год оборот 34 000 рублей, 2 000 000 кирпичей, 58 рабочих, 1 паровая машина в 4 силы, заведующий мещанин Георгий Салогуб.

В начале 20 века кирпичный завод принадлежал уже наследникам Виктора Субботина. В 1900 году завод произвел 6 миллионов кирпичей. На 1903-й есть сведения про завод на 85 рабочих по Большой Васильковской, принадлежащий Шталь Екатерине Викторовне (дочери Виктора Субботина и Эмилии Субботиной). Переделанный особняк Шталь стоит сейчас на Шелковичной, 12.\\
% На 1903 год там работало 149 человек.\\ 

\noindent\textbf{Э. САНДУКОВСКАГО} – Эйзер Меер Сандуковский, купец 1-й гильдии. Завод на Сырецкой, 25. 1903: 25 рабочих. Адрес на 1912 год прежний.

Известно также клеймо «Э. САНДУКОВСКАГО Ю.45», вероятно отсылающее к некоему адресу, например Юрковская 45.

Сандуковский умер во время Великой Отечественной войны где-то «в эвакуации», в возрасте 55 лет, «от старческой немощи». Если число верно, то даже если положим за год смерти 1945, получим год рождения 1890-й. И уже значится в списках купцов 1-й гильдии за 1908 год.\\

\noindent\textbf{Ю}\\

\noindent\textbf{ЮР. КЕРАМИЧЕСКИЙ ЗАВОД Б.Ф.} – возможно, Юрковицкий керамический завод Баронессы Фиркс. См. клеймо \textbf{Б.Ф.}\\

\noindent\textbf{Я.БЕРНЕРЪ} – Яков Николаевич Бернер (1838-1914, похоронен на Байковом), купец 2-й гильдии, пожалуй самый известный производитель киевского кирпича. На 1887, завод в Корчеватом (основан в 1881 году)\footnote{Прежде того, Бернер арендовал в Корчеватом кирпичный завод (на три обжигательные печи) у Выдубецкого монастыря, и заработав на этом, построил завод собственный.} – 1 500 000 кирпичей, оборот 20 000 рублей, 58 рабочих. 1894: завод на Большой Васильковской 176 (178? заведующий «Алек. Ник. Бернер») – его глинище было по месту нынешней Лыбедской площади (Бывшая усадьба Федора Чернышова) и сохранялось еще после Великой Отечественной войны\footnote{Описывая послевоенный Киев, Вилен Хацкевич в книге «Так говорил старик Ольшанский» вспоминает, что жители Демиевки купались тогда в озере под названием «Бернер» у бывшего кирпичного завода. Конечно же, то было заполненное водой глинище. Посередине под водой стояло два телеграфных столба. Высокий берег обрывался круто. Уже после войны в это глубокое, с мутной водой озеро свозили строительный мусор окрестные предприятия.}, там где ныне северная половина «Океан-Плаза», что через переулок от здания-тарелки, и часть станции метро Лыбедская; завод в Китаево (близ Мышеловки); завод в Тетиеве.

На 1894, для завода в Корчеватом указано: 1 500 000 кирпичей, оборот 15 000 рублей, 80 рабочих.

1890, завод в Китаево имел оборот 12 000 рублей, второй завод не указан вообще. На 1900 – 70 пришлых рабочих (неясно, на каком заводе). На 1903 – Большая Васильковская – 55 рабочих, Корчеватое – 140 рабочих. На 1911 – Большая Васильковская, 180 – 81 рабочий. Заводы в Демиевке и Мышеловке – 62 рабочих. На 1912 указаны заводы на Большой Васильковской, 180 и Кирилловской, 88. Последний – бывший братьев Зарембских. 

По материалам дела Бейлиса\cite{beylisdelo} очевидно, что уже в 1911 году «усадьба Бернера» была к северо-западу от глинища Зайцева, по соседству, напротив усадьбы на Кирилловской 88 (она на другой стороне улицы). Глинище Зарембских находилось в усадьбе по адресу Кирилловская, 69 (после 1898 года номер сдвинулся и стал 71), что совпадает с «усадьбой Бернера» из дела Бейлиса. А на 1915 год усадьба по адресу Кирилловская, 88 уже числится за «Алек. Ник. Бернером.», и это не сын Алексей, а вероятно брат.

1913: указан адрес Большая Васильковская, 180; производство 3 000 000 кирпичей, 60 рабочих, управляющий Капралов П. К..

Постоянный сбыт кирпичей Бернера обеспечивался во многом строительным подрядчиком, купцом 1-й гильдии Львом Гинзбургом.

Купец-старовер Михаил Парфентьевич Дегтерев (годы жизни: 1831-1898) завещал Киеву значительные деньги на строительство богадельни и приюта для бедных. В числе душеприказчиков был назван Бернер. Итогом этого решения были поставки для строительства кирпичей с заводов Бернера\footnote{Зная «корпоративность» старообрядцев, логично было ожидать от Дегтерева желания строить из кирпича, допустим, Булышкиных. В дело же вступили Бернер с неизменным подрядчиком Львом Гинзбургом. Подряд на  626 000 рублей – не фунт изюму. Гинзбург умер в 1926 году. Осип Мандельштам написал про него в своих воспоминаниях о Киеве: «Мученики частного капитала чтут память знаменитого подрядчика Гинзбурга, баснословного домовладельца, который умер нищим (киевляне любят сильные выражения) в советской больнице».}. 

Кирпичами вместо денег Бернер иногда делал благотворительные пожертвования. Однако на склоне лет, в завещании он отписал 100 000 рублей «в ведение Киевского городского общественного управления, для постройки в г. Киеве, если можно, в усадьбе Александровской больницы, особого капитального здания для содержания городским управлением хронических больных из беднейших постоянных жителей города Киева» – однако сие не было воплощено в жизнь по неведомой для меня причине.\\

\noindent\textbf{ЯБЛОНКА} – завод Бенита и Валета Сагатовских в селе Яблонка (теперь это часть Бучи, микрорайон Яблонька, в который объединились селения Яблонка, Мельники, Лесная Буча и Ястремщин). 

По меньшей мере с 1860-х, Яблонька была имением католика Северина Осиповича Сагатовского. Его кирпичный завод известен в 1876-1887 годах (по некоторым данным, завод построен в 1868-м). На 1887 год сведения по нему: 1 000 000 кирпичей, оборот 10 300 рублей, 5 рабочих. Также на 1887 год Похилевич пишет\cite{pohyluezd}, что у Сагатовского два кирпичных завода – «при деревне и хуторе Рудне, оцениваемых казною в 800 рублей».

На 1900 год Сагатовские заводы произвели 3 миллиона штук кирпича.

Бенит и Валет Сагатовские – полагаю, наследники Сагатовского и завод тот же самый. На 1913 год владельцами указаны «наследники Сагатовского Ф.С.».% Кстати существует кирпичи с клеймом «БNВ» (N маленькая), которое очень здорово подошло бы к Бениту и Валету, если бы не известное клеймо «ЯБЛОНКА».

На 1894 год арендатор завода – «Ив. Мак. Калашников». На 1903 на заводе 90 рабочих, паровая машина в 100 сил. 1913: заведующий Шимчик А. С., паровая машина в 120 лошадиных сил, производство 5 000 000 кирпичей, 140 рабочих.

На кирпичном заводе Сагатовских, когда он возобновил работу в 1926 году (отдельный завод в Ирпене в то время не был еще восстановлен), работал писатель Николай Носов, о чем он рассказывает в своей книге «Тайна на дне колодца». В советское время завод стал известен как Бучанский кирпичный завод (БКЗ)\footnote{ 50°32'12"N 30°14'10"E}, в 1964 году реорганизованный в Бучанское заводоуправление строительных материалов, а затем в Ирпенский комбинат «Победа». 

На 1957 год БКЗ производил 100 000 кирпичей в сутки.

В 1978 году при разработке одного из карьеров БКЗ были найдены череп и кости шерстистого носорога (на глубине 15 метров), а также янтарь.

На 2016, БКЗ – заброшен. Рядом – жилмассив «БКЗ» Известны также клейма «БКЗ».

Рядом, на восток, с БКЗ соседствует Ирпенский комбинат стеновых материалов и строительных пластмасс «Прогресс», также выпускающий кирпичи. Он примыкает к двум большим карьерным озерам, северное из коих называют «Земснаряд»\footnote{50°32′38″N 30°15′11″E} по затонувшему там земснаряду. 

«Прогресс» возник так. В 1955 году было создано Ирпенское заводоуправление кирпично-черепичных заводов. Собственно это и был сам завод. За два последующих года он произвел 78 миллионов кирпичей, а с 1964 года освоил производство продукции их полистирола и пенопласта. Позже завод получил название «Прогресс».

На 2007 год он имел мощности выпуска 100 миллионов кирпичей в год, но производил 30 миллионов. На 2018 территория завода частично занята другими компаниями – как строительными, так и например по выпуску тортов, и четырьмя общежитиями, оставшимися со времен расцвета Прогресса», таким образом представляя собой как бы самостоятельный городок на стыке Бучи и Ирпеня.\\ 

\noindent\textbf{ЯБ. и Ко}\\

\noindent\textbf{Я. КОЗИНСКІЙ} – Яков Бенцианович Козинский, на 1912 завод по адресу Копыловская, 63. 1913: Копыловская, 51. 1903: 90 рабочих.

Козинский погиб в Бабьем яру, в возрасте 73 года, вместе с женой Бертой и дочерью Елизаветой. На 1941 год проживали на Михайловской, 17.\\

\noindent\textbf{Я.Ф.}\\  

\noindent\textbf{SILBERMAN} – Я. Зильберман.\\

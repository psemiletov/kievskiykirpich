\chapter*{Предварение}
\markboth{\MakeUppercase{Предварение}}{}
\addcontentsline{toc}{chapter}{Предварение}

Слово «кирпич» обычно считают тюркским. Казахи произносят – кирпиш, киргизы – киш. Однако население Индии, что общается на хинди и тамильском языках, говорят как мы – кирпич. Может в этом намек на разгадку? В Западной Европе, среди славян и не только, больше ходили производные от слова «цегла». Цеглой называли кирпич на «роськой мове» – официальном языке Великого княжества Литовского, цигелем величают кирпич немцы.

В летописях тот древний кирпич, что мы следом за учеными кличем «плинфой», именовали просто «камнем». 

О каких-либо киевских кирпичных заводах документальные сведения появляются в 16 веке, сперва смутные. Обратимся к документам того времени, когда Киев входил в состав Великого княжества Литовского, включавшего в себя земли современной Литвы, Белоруси и Украины. Столицей княжества был город со славянским именем Вильнэ, Вильно (Вольное) – нынешний Вильнюс.
 
В описании Киева и Киевского замка за 1545 год королевскими люстраторами сказано\cite{sbornikmat}:

\begin{quotation}
Цегла. Цеглы на потребу замкову зготованое, выпаленое перед местом\footnote{«Место» – город. Отсюда слово «мещане» – горожане.}, подле острогу подле Днепра накладена шопа\footnote{Сарай.} полна.
\end{quotation}

Где-то перед городом сделали кирпич-сырец, обожгли его (выпалили) и затем сложили в сарай возле острога около Днепра.

В начале 17 века, кирпичный завод всплывает касательно землевладений земянина\footnote{Воинское сословие, наделяемое за службу землей.} Войтеха Соколовского, которому Киевский воевода, князь Константин Острожский, дал во владение земли, в том числе «у потока церкви святого Миколы Ерданскего» – это на Кирилловских высотах\footnote{Цепь холмов от Подола до Куреневки, знаменитая пещерами, древнейшими захоронениями и скопищем костей более 60 мамонтов, получившем название Кирилловской стоянки.} у Иорданской церкви. Соколовский построил там кирпичный завод – цегельную шопу, которую разорили люди, посланные игумном Кирилловского монастыря, заявившего свои права на ту же землю. 

Выдержка из судебного документа 1609 года о разбирательстве между Соколовским да игумном Кирилловского монастыря Василием Красовским, где приводится жалоба Соколовского на игумна, который\cite{akty}:

\begin{quotation}
наславши моцно кгвалтом слуг своих Андрейка и Богдана и иных многих з ними, которых вы назвиска добре ведаете и оных называете, на кгрунт его власный близко двора его перегородья шопу, для робенья цеглы збудованую, побрать и разметать росказали и цеглы в ней десять тысячей до выпаленья выготованое порозбивали и в нивечь обернули, которая шопа коштовала двадцать коп и пять коп Литовских, а цеглы каждая тисеча коштовала по две копы Литовских; там же подле шопы печь мурованый дошками побитый спалили и печь порозваливали, который печь коштовал коп петнадцать грошей Литовских [...]
\end{quotation}

Посланные игумном слуги его – Андрейко, Богдан и прочие, совершили нападение на кгрунт (усадьбу или земельный надел) Соколовского близ его двора, где была сооружена шопа для производства кирпича. Там лежало 10 000 штук кирпича-сырца, ожидавшего обжига. Кирпич разбили, шопу разорили, печь развалили. Из документа узнаём и тогдашние цены. 

Позже Кирилловский монастырь завел на Кирилловских высотах собственный кирпичный завод.

%Позднейший кирпичный завод самого Кирилловского монастыря работал где-то поблизости, по одним данным

%известен на ближайшем участке, позже перешедшем к дворянам Гудим-Левковичам – затем там «прописался» завод Рихерта, на Нижнеюрковской 2.

%Завод Рихерта расположен неподалеку от Иорданской церкви. Некий кирпичный завод примыкал к усадьбе деревянной Иорданской Николаевской церкви по 1826 год, когда завод погорел и спалил заодно храм. Для богослужений тогда стали использовать трапезную Дмитриевскую, а затем отстроили одноименную кирпичную церковь.. Кирпичный завод, дальнейшая судьба коего от меня ускользает, находился примерно по нынешнему адресу на Кирилловской, 41.

А генеральный проповедник киевского доминиканского монастыря Петр Развидовский в своих записках, относящихся к 1634-1664 годам, упоминает\cite{sbornikmat} принадлежащий бернардинам «грунт за замком высоким киевским (в старом Киеве), где мы кирпичный завод поставили. Около того на нашем грунте сидело несколько кожемяков, которые с тех грунтов платили чинш» – возможно, речь идет об урочище Кожемяки, однако несколько сбивает с толку уточнение «в старом Киеве», коим именовали Гору, верхнюю часть города. Кожемяки удобны для расположения завода потому, что там протекал ручей Киянка, ныне журчащий в коллекторе\footnote{В 2015-16 годах часть вод оной Киянки питала небольшое, даже с камышом, болотце посреди стройки в жилищном комплексе «Воздвиженка».}.

В 1701 году по документам проходят кирпичные заводы (цегельни), построенные мещанами Киева под горой Щекавицей и разоряемые представителями Кирилловского монастыря. Тогдашняя межевая комиссия постановила, что цегельни переходят во владение Киевского магистрата\cite[том 2, стр. 22]{akty}:

\begin{quotation}
А прето, яко неслушне тое деялося, же отцеве Кириловские цегелень, чрез мещан Киевских под горою Шкавицею построенных, им, мещаном, забороняли и оныя разоряли, так мы, вышей реченныи особы, присужаем, абы от сего часу, тыи цегельне, и все тое подгорье, под горою Шкавицею по поток будучое, майстрату и всем мещанам Киевским вечне належало [...]
\end{quotation}

%В «Ведомости киевского магистрата Киева нижнего города имянуемого Подола, о мещанах, купечеством промышляючих, ремесленных и о протчем, 1755 года марта» упоминаются невесть чьи «корпичные заводы»:

%В городе Киевоподоле имеются казенной шелковой и партикулярные кожевные и корпичные заводы.

История знаменитого киевского желтого кирпича, получаемого из синей глины, разворачивается в девятнадцатом веке и начале двадцатого. Этому-то кирпичу, одушевленному клеймами, и посвящена большей частью сия книжка.

Вначале она задумывалась как приложение к четвертой редакции моей краеведческой книги «Ересь о Киеве», а упорядоченный материал использовался внутренне для рабочих нужд. Однако собранные сведения переросли в отдельное произведение.

Книга построена как справочник. В главу о клеймах я поместил известные мне клейма и трактовки оных. Некоторые поныне остаются загадкой, толкование других не бесспорно и подлежит возможному исправлению.

В главе о производителях вы прочтете о кирпичных заводах, про которые мне известны некие данные, кроме клейма, а также о заводах казенных и лаврском. 

Глава про карты также послужит подспорьем в определении связей между положением завода, его владельцем и клеймом. Все три главы дополняют друг друга.

В конце книги я поместил картинки, относящиеся к киевским кирпичным заводам, а также выдержки из двух дореволюционных статей, посвященных кирпичной синей глине.

Благодарю моих друзей, неизменных спутников и спутниц по краеведческим вылазкам – Колю Арестова, Алину Шиндировскую – за прямое или косвенное влияние на создание книги, а также родителей за обсуждение оной.

В работе над второй редакцией чрезвычайно полезным было общение с исследователем Бобруйской крепости Валерием Мельниковым, который основательно занялся вопросом клеймления кирпичей на казенных заводах.

Во второй редакции, мне кажется, появился второй слой повествования – за сухими справочными сведениями, в примечаниях проступили частички судеб живых людей, владельцев кирпичных заводов. Хотя это не их руки делали кирпичи. Имена рабочих, кроме писателя Носова, остались где-то там в прошлом, не вытащишь. 

Да и о большинстве заводчиков-кирпичников известно мало. Я не ставил впрочем целью приводить их жизнеописания, даже если знаю. По тем заводчикам, которые достаточно освещены в истории, я даю краткие сведения, выжимку. По тем же, о ком сами эти сведения сохранились лишь отрывочные, и привожу эти самые отрывки – быть может, это поможет кому-то в дальнейших изысканиях.

Третья редакция книги, помимо исправления ошибок, существенно дополнена сведениями о производителях кирпича и картинками.

К работе над четвертой редакцией меня сподвиг Максим Коляда, приславший индекс заметок про кирпичное дело к дореволюционной газете «Киевлянин». Обогатив справочник сведениями из «Киевлянина», я дополнил текст жизнеописательными данными и заодно сделал его более удобочитаемым, «разгрузив» примечания. Также добавил кое-какие иллюстрации.

В пятой редакции я подправил текст и добавил новые сведения про завод купца Терехова.

Текст книги набран и сверстан под Линуксом в редакторе \href{http://semiletov.org/tea}{TEA}, собственной разработки. Вёрстка подготовлена при помощи удивительного средства вёрстки \href{http://www.latex-project.org}{\LaTeX} с движком Lua\TeX~. Я использовал семейство свободных шрифтов DejaVu.

Новые редакции книги, по мере их выхода, будут выкладываться по следующим адресам:\\ 

\noindent
\href{http://semiletov.org}{semiletov.org} (мой сайт)\\ 
\href{https://www.facebook.com/groups/kievograd/}{www.facebook.com/groups/kievograd/} (группа Киевоград в FB)\\ 
\href{https://vk.com/eresokieve}{vk.com/eresokieve} (группа книги «Ересь о Киеве» в Контакте)\\ 
\href{http://semiletov.org/kievograd}{http://semiletov.org/kievograd} (проект неформального краеведения Киевоград)\\ 

Для связи со мной:\\ \href{mailto:peter.semiletov@gmail.com}{Мыло –  peter.semiletov@gmail.com}\\ 
\href{https://t.me/petrsemiletov}{Телега – https://t.me/petrsemiletov}\\
\href{https://vk.com/id10275972}{ВК – vk.com/id10275972}\\
\href{https://www.facebook.com/peter.semiletov}{ФБ – www.facebook.com/peter.semiletov}\\


Буду рад отзывам!
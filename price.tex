\chapter*{Цены на кирпич}
\addcontentsline{toc}{chapter}{Цены на кирпич}

Некоторые примечания.

Летний строительный сезон делился на две части – до дня Петра и Павла (29 июня по старому стилю) и после. Во второй части кирпичное производство сокращалось, а некоторые заводы вообще приостанавливали производство.

В самом конце 19 века, когда в Киеве была «строительная горячка» (при помощи банковских ипотечных кредитов), кирпич продавался по 27-28 рублей за тысячу. Возникали новые заводы, а старые увеличивали объемы производства. 

Затем строительный бум прекратился, спрос на кирпичи сократился, и цены стали падать.  В начале 20 века начался так называемый кирпичный кризис.

Далее будут числа, для справки же сообщу еще, для оценки объемов продаж – на постройку здания шло не менее нескольких сотен тысяч штук кирпичей. А годовая подписка газеты «Киевлянин» в 1901 году стоила 12 рублей.

1899 – 15-16 рублей за 1000, 20 рублей с доставкой

1900 – зимой 19-21 рубль за 1000 штук, с доставкой на место. Летом стоимость упала до 17 рублей, осенью до 16. Объем производства 200 миллионов штук (сопоставимо с «докризисным» объемом).

1901, зима – 17-18 рублей за тысячу, апрель 16,75 рублей, июнь 14,50, июль 15-16, ноябрь 14-15. 
Объем производства около 80 миллионов штук.

1902, январь – 13 рублей (но его не покупают, считают дорогим).
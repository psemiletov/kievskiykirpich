\chapter{Производители}

В этой главе я помещаю производителей заводов, чьих клейм не знаю или трактовка коих неоднозначна. 

Сведения о кирпичных складах, устроенных в начале 20 века многими производителями и перекупщиками на Подоле вдоль набережной, опускаю – а таковые склады были у Бялика, Волкова, Кирьянова, Василия Ивановича, Мозгового И. Ф., Коблова, Чернояровых, Шпрунга, Снежко. Эти склады купно с конторами вносят лишнюю путаницу, и только по адресу можно понять, что это не очередной завод, а его представительство, где можно приобрести либо заказать партию кирпича, подвезенного по реке. Всё что в Гавани, на Подоле, или в центре города – это конторы, не заводы.\\

\noindent\textbf{Алексеев Егор Васильевич}, отставной майор – завод с 1880 года в Халепье. 1894: 200 000 кирпичей, оборот 2 000 рублей, 7 рабочих.\\

\noindent\textbf{Алексеева Апполинария Петровна} – завод в Протасовом Яру. 1903: 12 рабочих.\\

\noindent\textbf{Байдашников Тимофей} – некий завод, на 1900 год: 11 пришлых рабочих. Очевидна связь с Иваном Даниловичем Байдашниковым и Данилой Даниловичем Бойдашниковым.\\

\noindent\textbf{Бойдашников Данила Данилович}, крестьянин – завод по Житомирскому шоссе. 1894: 943 000 кирпичей, оборот 9 000 рублей, 30 рабочих.\\

\noindent\textbf{Батухин} – на плане Киева 1879 года показан «кирпичный завод Батухина» у северной стороны Протасова яра, первый по счету от его устья. Возможно, сейчас там переулок Докучаевский. На той карте еще дальше от устья показан еще один «кирпичный завод».\\


\noindent\textbf{Белан Павел} – завод в Василькове.\\

\noindent\textbf{Бернер Ал. Ник.} – на 1915 указан завод на Кирилловской, 88 – адрес завода братьев Зарембских. Это Бернер не Яков, но «Ал. Ник». Про Якова Николаевича см. раздел «Клейма». У Якова Бернера был сын Алексей, но «Ал. Ник» – не сын. Может тогда брат? «Алек. Ник. Бернер» числится в 1894 году заведующим заводом Якова Бернера на Большой Васильковской.\\

%\textbf{\textbf{Бергенгейм Э. Э.} – 1909: огнеупорный кирпич, адрес указан: Большая Васильковская, 30 – стало быть там не завод, а контора.} 

\noindent\textbf{Браницкая Марья Евстафьевна}, графиня – завод в Триполье с 1858 года. 1894: \mbox{400 000} кирпичей, оборот \mbox{4 000}  рублей, 25 рабочих.

Известно одно из кирпичных клейм Браницких – четырехзначный год выпуска, где после первых двух цифр стоит герб Браницких в виде щита с тремя горизонтальными палочками. Однако насколько это клеймо соотносится с заводом в Триполье, я не знаю, ведь у Браницких были кирпичные заводы и в других имениях: Ставищах, Гребенках Васильковского уезда (12 рабочих вероятно в 1900 году, принадлежал той же Марье Браницкой), Кожанке Васильковского уезда (построен Марьей Браницкой). Кроме того, в 1830-м Александрой Васильевной Браницкой был устроен завод в Белой Церкви.\\

\noindent\textbf{Бородавка Петр}, козак Киевской сотни – завод на Кирилловских высотах или окрестностях.

Завод существовал по крайней мере между 1709-1732 годами, и упомянут в жалобе козака митрополиту Заборовскому 10 февраля 1732 года, где сказано, что Бородавка вместо чинша (погодовой арендной платы) давал Кирилловскому монастырю кирпич на обустройство церкви («дати цегли паленой доброй на вибуркованя церкви нарубь цеглою»), и было выдавно 25 000 штук, а затем еще 500 на монастырские нужды. 
%pohilmon 
Церковь было «вибурковано», Бородавку должны были освободить от выплаты чинша пожизненно, однако новые игумны продолжали брать с козака кирпич – в 1720-26 годах 6 000, в 1720-32 – 26 500. Игумен Феодосий, ко времени коего относятся последние годы, не только взял кирпичи, но и приказал слугам своим напасть на владения Бородавки.

При разбирательстве Феодосий утверждал, что Бородавка «ради спасения своего дал цеглы до монастыря на вымощение церкви колько было потребно».\\

\noindent\textbf{Борышпольский} – завод около Стаек, остановил работу в 1902 году.\\

\noindent\textbf{Вайсберг Азрим} – завод в Василькове.\\

\noindent\textbf{Василенко Макар Григорьевич}, крестьянин – в 1894 году известен завод на Демиевке, основанный в 1846-м. Василенко – арендатор еще одного завода, у чиновника Алексея Гетманова.\\

\noindent\textbf{Вощинкин Василий Кузьмич}, мещанин – завод с 1871 года в селе Витачеве. 1894: 420 000 кирпичей, оборот \mbox{5 000} рублей, 15 рабочих.\\

\noindent\textbf{Выдубицкий монастырь}

В 1713 году из-под пера иеромонаха Григория, наместника митрополита, вышло следующее свидетельство:

\begin{quotation}
С копиї копия. Я, нижей подписанний, извістно творю сим писанием, иж ясне в Бгу преосвященний кир Иоасаф Кроковский, митрополит киевский, галицкий и всея Малия Россиї, упросивши у всечестнаго отца Лаврентия Горки, за відомом всей братиї Видубыцкой обытели місца, на их власних кгрунтах между Козовищем и Либедю лежащаго, веліл нам на том місці цегельню построїти, яко уже и построена. При которой и другая имілась строїти на то кгрунті власном видубицком в року 1713 мсця мая дня 4, на що для лучшаго свыдітелства сие писмо далось монастиреві Видубицкому за печатю меншою сто Софійскою.

В подлинном подпись таков:
Иеромонах Григорий, намісник катедралный сто Софійский митрополитанский киевский
\end{quotation}


Согласно «Описанию Посполитым слободи Зверинця в Уезде киевском состоящого, учиненногое 1786 года», среди описаний владений Выдубицкого монастыря, находим такое:

\begin{quotation}
2. Вне отдаленности от монастыря надрекою днепром при урочище Козовице имеется кирпичный монастырский завод, 

из которой часть една уступлена киевской казенной палате по контракту усего прилагаемого подлитерою А. 

Другая часть заведеная Киевским купцу Михайле Григаренку и мещанину Тимофее Заренбе по контракту у сего прилагаемого подлитерою В [перечень построек и предметов в них].

Третая занята армизоною командою безвсякого оной отмонастиря от воду и Позволения,

а четвертая часть бывшая около двадцяти лет взаведовании Киевософейского кафодрального монастиря обратно сему монастырю отдано [перечень].

Там же на Либеди между кирпичными артилерийскими заводами близ мельнички в обивательской избе Григория Телички продается только ценная горелка из оного обывателя монастирской посуды [...]
\end{quotation}

О заводе в примечаниях сказано: «за содержание сего завода получал монастирь з казни вгод по 60 ру.», а про «другую часть» такое: «за содержание сего завода от означенных откупцов получаемо было вгод денег по 120 рублей». 

В 1860 году известен завод в Мышеловке. На конец 19 века – в Корчеватом (его арендовал Бернер), однако по смежности местностей не исключено, что речь идет об одном и том же заводе. На конец 19 века, в Корчеватом у монастыря была дача в 26 десятин, с кирпичным заводом о трех печах. Похилевич пишет, что завод большей частью сдавался в аренду, «рублей за 300».\\

\noindent\textbf{Гетманов Алексей} – на 1894 год завод на Демиевке.\\

\noindent\textbf{Гилевич и Ко} – на стыке 19-20 веков деятельный инженер путей сообщения А.Т. Гилевич затеял производство в Киеве силикатного кирпича (который дешевле глиняного, но хуже переносит высокие температуры, лучше впитывает сырость, и непригоден для постройки многоэтажных зданий). Дескать, засыпем дешевым силикатным кирпичом весь город, и производители обычного кирпича тоже скинут цены.

В 1899 году Гилевич настойчиво пытался пробить ходатайство, подавая его городскому голове, об отводе под завод участка на Трухановом острове, а сырье (песок) собирался брать на днепровских отмелях «от Оболонского залива до Цепного моста». Под завод он также соглашался взять землю хоть не на Трухановом, однако в урочище Наталке, и собирался давать по 20-25 миллионов кирпичей в год.

Летом 1900 года завод таки открылся, в усадьбе митрополитанского дома, на углу Жилянской и Тарасовской улиц. Усадьба занимала 13 десятин вдоль Лыбеди, и была взята в аренду на 20 лет.

Завод строился на деньги товарищества, состоящего из Гилевича, Титаренко и Кисинского. Капитал товарищества составил 50 000 рублей. 

Завод представлял собой два деревянных здания с германскими машинами, в том числе паровой на 40 сил. Для приведения ее в движение, а также для превращения силикатного кирпича в твердый при помощи пара, установили паровой водотрубный котел в 120 сил.

На время открытия завода сообщалось, что он будет выпускать по 20 000 силикатных кирпичей в день, в год же – 6 миллионов.

Но вместо гарантированных 20 лет работы завода, уже в 1902 году его участок начали отчуждать под товарную и сортировочную станцию Юго-западных железных дорог, а завод едва работал. Возникшие в промежутке с 1900 по 1902 еще два завода силикатного кирпича прогорели – один (около Триумфальных ворот) в переносном смысле, другой (на левом берегу, близ железнодорожного моста, принадлежавший товариществу Геннеман, Генри Смит и Ко) – сгорел буквально.\\


\noindent\textbf{Голятовская Анна Ивановна} – в 1903 году, завод на Нижнеюрковской, 16. 1894: арендатор Иван Федорович Никитин.\\

\noindent\textbf{Горенко Марина Клементьевна}, купчиха – завод в селе Старо-Петровском с 1877 года. 1894: 550 000 кирпичей, оборот 7 000 рублей, 25 рабочих. В 1909-10 годах некий Горенко С. Т. торговал кирпичами в Гавани, быть может наследник.\\

\noindent\textbf{Горенштейн Арнольд Исаевич}, австрийский подданный, потомственный почетный гражданин – завод на Демиевке с 1860 или 1880 года. Арендаторы на 1894: «Ант. Вильг. Папст», чиновник Алексей Гетманов. 1887: 400 000 кирпичей в год, оборот 5 000 рублей, 16 рабочих. 1894: 350 000 кирпичей, оборот 3 000 рублей, 17 рабочих.\\

\noindent\textbf{Гохлернер Эвель (Евель) Михель}, купец – завод в селе Мостище Старопетровской волости. 1903: 30 рабочих.\\

\noindent\textbf{Григорович Иван}, киевский мещанин и мурового дела мастер (больше известен как архитектор Григорович-Барский, 1713-1791) – на карте 1752 года на перекрестке нынешних Нижнеюрковской и Кирилловской (Фрунзе) показаны «Кирпичные заводы Мещанина Ивана Григоровича, который речку Юрковицу принял на одну сажень». Умозрительно место заводов соответствует будущему заводу Гудим-Левковичей и Рихерта, и старинному заводу Кирилловского монастыря. По документам Кирилловского монастыря, Григорович Иван, Иосиф Гудима, Михаил Гудима арендовали там у монастыря, по чиншу, земельные участки.

По документам, в 1770-х чинш Григоровича увеличился до 10 рублей в год за постройку завода, но мы знаем по карте 1752 года, о неких заводах Григоровича. Быть может, прежде он арендовал монастырскую «цегельню», а потом обзавелся своей?\\

\noindent\textbf{Губкин Ф. Я.} – 1910: указан адрес Сырецкая, 7.\\

\noindent\textbf{Гудим-Левковичи} – на месте их завода действовал потом завод Михаила Рихерта по адресу Нижнеюрковская, 2. 

Время приобретения Гудимами завода – 1765 год – перекликается с секуляризацией церковных земель манифестом 1764 года Екатерины II, отнявшим у церкви большинство её землевладений. Раньше я полагал, что Гудимы приобрели завод Кирилловского монастыря, стоявший примерно там же, но существуют данные о монастырской цегельне существовала по крайней мере в 1779-1780 годах. В тему читайте еще про завод Ивана Григоровича.

В «Статистическом описании Киевской губернии»\cite{fundstat} 1852 года издания дается подробное описание завода богатой дворянской семьи землевладельцев Гудим-Левковичей:

\begin{quotation}
Кирпично-кафельный завод в Киеве г. Гудим-Левковича – старинное заведение, существовавшее еще с 1765 году в незначительном размере, в Плоской части, у подножия гор\footnote{Кирилловские высоты. Гора, у которой стоял завод, раньше называлась Лысой, а позже на нее с соседнего отрога, где кладбище старообрядцев, перешло название «Юрковица».}, которых нижний пласт состоит здесь из так называемой синей глины, способной для выделки печного кирпича.

Завод расширен и улучшен только в недавнее время, в 1842 г., и состоит теперь из следующих частей, назначенных собственно для кирпичного производства. Обжигательная печь, устройство которой стоило 900 рублей серебром, два сарая для просушки сырого кирпича – 200 рублей серебром, принадлежности заводския, как-то: столы, формы, тачки, заступы и т.п. – 75 рублей серебром. Всего на постройку употреблено 1 175 рублей серебром.

Глина для кирпича добывается при самом заводе из горы, принадлежащей его владельцу; на выемку ея и перевозку к сараям в тачках употребляются поденьщики, а постоянных кирпичников – 4 человека, выделывающих, в двух формах, каждая пара по 500 штук кирпича в день. Кирпичедельцы работают от штуки и получают по 1 рублю серебром за 1 000 приготовленных кирпичей. 

На укладку сырого кирпича в сараи, загрузку его в обжигательные печи и выгрузку нанимаются поденьщики. На обжиг кирпича выходит 50 саженей дров, по 8 рублей серебром за сажень, на 400 рублей серебром.

Годовая выделка обозженнаго не превышает 100 000 штук; все это количество раскупается на самом заводе, ценою по 10 рублей серебром за тысячу.
\end{quotation}

На этом же заводе производились и кафели, глину для простых без глазури кафель добывали тут же в горе, а для глазурных белых завозили глину из-под Глухова и Межигорья.\\

\noindent\textbf{Гусева Екатерина Гавриловна} – на 1911 год владела заводом на Копыловская, 61 – это бывший завод Булышкиной. Гусева – наследница Булышкиной. 1910: арендатор купец «Козинский Як. Бенц.», 3 000 000 кирпичей, оборот 36 000 или 42 000 рублей, 100 рабочих. 1911: 80 рабочих.\\

\noindent\textbf{Дарницкий комбинат строительных материалов и конструкций} – известен в Никольской слободке на берегу Черторыи, на улицах Комбинатной и Сагайдака. На 2016 год земельные участки, где велось производство, застраиваются жилыми домами. Под указанным названием зарегистрирован в 1994, но существовал много раньше, по крайней мере еще в 1939-м, когда от Никольской Слободки до завода была служебно-грузовая линия бензотрамвая.\\

\noindent\textbf{Данилевский, Микута, Лонский А., Зелинский С., Минут Герман Карлович} – на 1903 год записаны владельцами завода в Демиевке, на 110 рабочих. Не это ли Минут и Ко?

Про завод Минута газета Киевлянин в номере 310 за 1901 год сообщает: 

\begin{quotation}
Завод этот, ведя производство почти исключительно механически, с применением всех усовершенствований в технике кирпичного дела, является пока единственным в Киеве. Кроме обыкновенного строевого кирпича, формовка которого производится, однако, машинным способом, завод выпускает на рынок фасонный\footnote{Фасонный кирпич в 1901 году был освоен и заводом наследников Субботина.} и радиальный кирпич. Последний из них идет исключительно на постройку дымовых труб.\end{quotation}

На карте 1914 года завод показан по месту современного адреса Голосеевская улица, 9.\\

\noindent\textbf{Дзегановский Петр Афанасьевич}, титулярный советник, дворянин – завод с 1865 года по адресу Протасов Яр, дом №42. На 1894 год сведения об арендаторе противоречивые – Игнат Васильевич Проценко, либо вахмистр Тимофей Фокин. 1894: 600 000 кирпичей, оборот 6 000 рублей, 21 рабочий. В 1914 году усадьба с домом по этому адресу была передана сыном Петра Афанасьевича, преподавателем кафедры химии в киевском универе св. Владимира, Александром Дзегановским, в собственность городу для открытия одноклассного приходского училища №72, имени А.П. и О.И.Дзегановских. Занимался благотворительностью и сам Петр Дзегановский.

А на балконе его дома по адресу Лютеранская, 13, в 2016 году есть кованые витиеватые перила с изображением буквы Д, подобной знаку футбольной команды «Динамо».

Дзегановский имел 10 детей и служил чиновником в Киевском магистрате.\\

\noindent\textbf{Довгонянский} – в 1887 году владел заводом в Халепье, что оценивался в 400 рублей.\\

\noindent\textbf{Долинские}, дворяне – завод, известный с 1850 года, по адресу Косогорный переулок, 10 (отходит вверх от Глубочицкой на гору Кудрявца). 1887: 600 000 кирпичей, оборот 7 000 рублей, 21 рабочий.\\

\noindent\textbf{Доломакин Тарас Адрианович} – завод в Мышеловке. 1903: 96 рабочих.\\

\noindent\textbf{Дынник Евгений Ефимович} – Нижне-Юрковская 53, завод учрежден в 1873 или 1878 году. 1887: 400 000 кирпичей, оборот 5 000, 16 рабочих. 1894: 500 000 кирпичей, оборот 5 000, 16 рабочих.\\

\noindent\textbf{Журавлев Максим Васильевич} – на 1911 год сведения о заводе на Большой Васильковской, 141: 60 рабочих. На тот же год, владелицей усадьбы по тому же адресу указана «Журавлева Мар. Вас»\footnote{Кстати, по адресу Большая Васильковская, 71 был дом Журавлева Макара Васильевича, который сдавал извозчикам напрокат лошадей с фаэтонами.}. Завод Журавлева на карте 1914 года показан, употребляя современные ориентиры, непосредственно на юго-запад от современного озера Глинка, на углу между Железнодорожным шоссе и бульваром Дружбы Народов, там где разворот эстакады и АЗС. Координаты: 50.40913879550456, 30.525978489072127\\

\noindent\textbf{Журавлев И. М.} – в 1912 году указан владельцем завода на Большой Васильковской, 141.\\

\noindent\textbf{Завитневич} – на 1911 год указаны завод в Демиевке, завод в Мышеловке. 1913: завод в Мышеловке.\\

\noindent\textbf{Залесский Станислав}, граф – завод в Рудом Селе (Володарский район Киевской области), известен в 1825 году. Наследница графа – племянница Фелициана Ивановская.\\

\noindent\textbf{Зелинского наследники} – завод в Пирогово с 1872 года. 1887: 1 000 000 кирпичей, оборот 15 000 рублей, 60 рабочих.\\

\noindent\textbf{Иозефат (Иоасаф) Ант. Андржеиовский (Андржеиевский)}, австрийский подданный – кафельный завод с 1882 года, Кирилловская, 62. 1894: оборот 23 000 рублей, 30 рабочих. 1911: 52 рабочих.\\

\noindent\textbf{Казенные заводы} – примеры клейм: \textbf{К. 1841}, \textbf{К 1846}, \textbf{К 1848}, \textbf{К. 18.47}, \textbf{КВ. 50}, \textbf{КВ 1846}, \textbf{К.В. 50}, \textbf{КВ 56}, \textbf{КВ 59}, \textbf{К.В. 60}, \textbf{К}, \textbf{КИК}, \textbf{К.И.К.}).

Тема клейм казенных, то бишь государственных заводов Российской Империи – штука запутанная, поскольку было несколько ведомств, каждое со своими заводами. Кроме того, одни ведомства преобразовывались в другие, а некоторые заводы, кажется, меняли хозяев, переходя от ведомства к ведомству. Поэтому мне, чтобы не запутать всё еще больше, остается лишь повествовать и не пробовать толковать противоречия.

Основными потребителями казенного кирпича в Киеве были Старо-Киевская, Печерская и Новая Печерская крепости\footnote{При необходимости закупался кирпич и у частных заводов.}. В 1730 году первые две крепости числились за 3-м Департаментом штатных крепостей.

В 1797 году, при строительстве Арсенала (рядом с Лаврой), киевский купец Михайло Иванов сын Григоренко, подряжаясь на работы, временно на четыре года принял Казенный артиллерийский кирпичный завод, для производства кирпича на возведение Арсенала

\begin{quotation}
Лета тысяча семьсот девяносто седьмого, ноября 17-го дня, киевской купец Михайло Иванов сын Григоренко заговорился в присутствии его сиятельства высокоповелительного генерал-фельдмаршала и разных орденов кавалера графа Ивана Петровича Салтыкова и господ инженер-генерал-лейтенанта и кавалера де Шардона, и от артиллерии генерал-майора и кавалера Мамонтова, окончить строением Киевский каменный Арсенал, на нижеследующих кондициях.[...]

ЧЕТВЕРТОЕ. Кирпич должен употребляем быть сделанный из хорошо вымятой глины, порядочно обозженной же красной, и половинчатого, чтоб не более было как на тысячу, 25 кирпичей, мера же его длиною шесть вершков, шириною три, толщиною один с четвертью вершок. [...]

ВОСЕМНАДЦАТОЕ. Казенный артиллерийский кирпичный завод на котором производилось дело казенного кирпича для Арсенала принять мне Григоренку совсем имеющимся на нем строением и инструментом, который исправить мне починкою в годность и по прошествии четырех лет времени отдать обратно в ведомство артиллерийское, не отбирая ничего починкою исправленного, и не требуя за то от казны денег, для делания же кирпича на оном заводе употреблять мне глину во все четыре года из самого того места и с коего брата была прежде казенными людьми.

В случае же невыполнения в чем-либо по сему контракту, взыскать со штрафом по законам на мне и на поручателях. По мне подлинное подписано к сему контракту подписал киевской купец Михайло Иванов сын Григоренко. В точном и исправном по содержанию сего контракта выполнении ручаемся, а в случае какой-либо неисправности обязуемся по точности оного за него Григоренка выполнить, в чем и подписуемся, киевский градской глава Георгий Рыбальский, киевской купец Ефим Митюк, киевской купец Козма Усович, киевской купец Федос Митюк.

Засвидетельствовал генерал-лейтенант фон Сухтелен

С копиею поверял регистратор Пензин
\end{quotation}

Между тем в те же годы строительством Киево-Печер\-ской цитадели занималась Киевская инженерная команда.

Тогда инженерные дела армии относились к \textbf{Артиллерийской Экспедиции}\footnote{Экспедиция – канцелярия, само учреждение, здание с военными чиновниками.}, ведавшей Инженерным департаментом. В 1802 году от Артиллерийской экспедиции отделили Инженерную Экспедицию и Инженерный Департамент. Значение последнего слова применительно к теме из году в год менялось – в 1802 подразумевалось «ведомство», в 1805 «окружное управление», в 1811 – уже «главное управление».

С 1811-12 годов учреждение Инженерную Экспедицию преобразовали в учреждение \textbf{Инженерный департамент}, сохранив прежние его функции – строительство и обеспечение крепостей, укреплений и прочих военных зданий, а также набор, обучение и учет состава инженерных частей, управлений и инженерных военно-учебных заведений.

При этом департаменте в 1856-1864 было \textbf{Управление казенных кирпичных заводов}, упраздненное после продажи заводов частникам. Функции же Инженерного департамента после 1862 года перешли к Главному инженерному управлению.

По крайней мере с 1832 по 1838 годы в Киеве существовало не менее двух казенных заводов некоего ведомства. Известно, что завод номер 1 тогда располагался под склонами с Лаврой, Печерской крепостью, между горой и Днепром. 

В 1832 году киевский военный губернатор Левашов писал во Временный комитет по благоустройству города, что заводы разрушают подножие склонов, и просил, «чтобы действие означенных заводов, как не в своем месте устроенных, теперь же было прекращено и чтобы поделанные ямы и рытвины были немедленно засыпаны». Но поскольку упразднение этих заводов сказалось бы на скорости строительства Главной Киевской крепости, заводы оставили, однако я не знаю, до какого времени.

Вероятно, клеймо \textbf{КИК} (некоторые толкуют его как «Киевская Инженерная Команда») относится к 1830-м, поскольку из такого кирпича построена в 1833 году Башня №4 (за зданием ЦИК).  Примечательно, что башня была построена напротив дома Ивана Федоровича Эйсмана, отца Густава Эйсмана. Дом этот был куплен старшим Эйсманом в 1818 году.

По статистике 1845 года, в Киеве насчитывалось 11 кирпичных заводов, из них казенных 2, монастырских 1, частных 8. Годовой оборот казенных и монастырских тогда составил 3 500 рублей серебром, частных – 34 000, на 200 рабочих и 6 мастеров.

На подробном военном плане, в разных вариантах хорошо известном в первой половине 19 века, и начале второй половины, в варианте за 1846 год обозначены казенные кирпичные заводы номер 1 и номер 2. Легенда к плану находится в первой части книги «Статистическое описание Киевской губернии», изданной в 1852 году Иваном Фундуклеем. 

По совокупности этих данных получается, что казенные кирпичные заводы с общим номером 1 располагались вдоль склона Бусовой горы – по Железнодорожному шоссе между низовьями нынешних улиц Тимирязевской и Киквидзе – а также на берегу между Днепром и южной оконечностью Зверинецкого холма. На том же плане, «кирпичные заводы номер 2» располагаются у Корчеватого.

Поскольку кирпичный завод номер 2 в Корчеватом точно был в ведомстве Инженерного департамента (на время продажи завода в 1862 году), логично предположить, что ему же принадлежал завод номер 1 около Бусовой горы.

В 1861 на казенных заводах Инженерного департамента трудились 3 арестантские роты (номера 25, 26 и 31, всего 592 человека). Заводы были упразднены в 1861 году, с продажей их. Два «корчеватских» завода в мае 1862 года выкупил, за 15 780 рублей, Адам Снежко.

В 1860-е годы известны также кирпичные заводы \textbf{Киевского приказа общественного призрения}, расположенные в Мышеловке. К этому же приказу относится, в шестидесятые же, кирпичный завод возле Караваевщины, частного владения мещан Головацких и городского выгона – речь идет о самой южной оконечности Зверинецкого холма, угол низовья Тимирязевской и Надднепрянского шоссе, а это часть кирпичного завода №1 с карты 1846 года.

Заводы у южного подножия Бусовой горы и Зверинецкого холма были упразднены, полагаю, при проведении там, под линией подножий гор, Киево-Курской железной дороги, концессия на постройку которой была дана в 1866 году. Уже к 1870-му рельсы проложили к Днепровскому мосту (ныне Дарницкий железнодорожно-автомобильный мост, «мост Кирпы»). Примерно в то же время на картах исчезают эти заводы, а на южном склоне горы Бусовицы появляется Госпитальное кладбище. Железная дорога, однако, не затронула кирпичный завод Лавры.\\

\noindent\textbf{Калашников Митр. Иванович} – завод в Корчеватом. 1903: 90 рабочих.\\

%\textbf{Канфер Г.Ш.} (1909) – завод в Гавани.\\

\noindent\textbf{Киево-Печерская Лавра} (\textbf{Успенская Лавра}) – клеймо предполагаю \textbf{КП число}, например \textbf{КП I848}.

Кирпичный завод в низовьях речки Лыбеди, около Выдубичей, документально прослеживается от времен, когда игуменьей Печерского женского монастаря (там сейчас Мыстецкий Арсенал) Марии Магдалины Мазепиной, матери гетмана Мазепы, понадобился кирпич. Привожу письмо:

\begin{quotation}
Року 1701, марта 17 дня. Известно твору, иж будучи игуминией монастыря Печерского девеческого, умыслила и взяла намерение, абы построена была церковь мурованная и предреченная той святой обители Вознесения Христова на небеса; до которого святолюбивого дела прислам их милости отцев монастыра Выдубицкого, абы позволили на грунте своем на Лыбеди цегелне ставити на уготованя цегли; речь слушную уваживши, его милость отец Варлаам Страховский, игумен монастыра Выдубицкого с отцами и братиею и свое цегельне готовие для поспиху позволити; те цегенльне по зготованью материи на церков вышереченную повынни будем благодарни монастыреве Выдубицкому отдати и и отци тоей святой обители яко своим власним грунтом владети мают. 

Известно, иж ради певности, на сем писании при печати монастырской руку пидписалам. Деялося в монастыру Печерском девическом Вознесении Христова, року, мисеца и дня выше написанного. Мария Магдалена Мазепиная, игумения Печерская Девического и Глуховского.\end{quotation}

То есть после использования заводов – цегельнь – Выдубицкого монастаря – на производство кирпича для церкви, Лавра должна была вернуть заводы Выдубицкому монастырю. 

Вероятнее всего, там же и появился, менее века спустя, основной лаврский кирпичный завод. Он располагался в принадлежащей монастырю местности подле устья Лыбеди, которая тогда впадала в Днепр чуть севернее зверинецкой Лысой горы, а именно между Лысой (Девич) горой и холмами Зверинца и Бусовицы. Местность и лаврское имение в ней слыли в документах под разными названиями, порой одновременно: Коноплянка, Нижняя Лыбедь, Верхняя Теличка, Зверинец.

Лаврентий Похилевич в книге 1865 года издания «Монастыри и церкви Киева»\cite{pohilmon} пишет:

\begin{quotation}
Коноплянка в выгонной черте города, по обеим сторонам речки Лыбеди. Урочище заключает в себе: собственно Коноплянку\footnote{Существовал еще хутор Коноплянка, там где сейчас радиовышка на Лысой горе.}, на правой стороне Лыбеди 136 десятин земли, поросшей большею частиею кустарником. Оно отведено в 1791 году для штатных лаврских служителей. б) Лаврские заводы кирпичные и пивоварней с прудом на Лыбеди, мельницею, садом и хозяйственными заведениями. Все эти заведения оценивались в 61 000 руб. и занимают пространства 86 десятин.
\end {quotation}

Глинище завода было на склоне Лысой горы. 
В 1859 году завод выпустил 1 000 000 кирпичей, \mbox{10 000} «кахеля» и 2 000 плошек.

В 1887 году мастер каменных работ из Варшавы,  Роберт Карстен (по другим источникам, А. И. Карсгенс), построил на заводе 16-камерную печь Гофмана, вместимостью по 1 000 кирпичей в камере.

1890: оборот 27 000 рублей, адрес указан «Нижняя Лыбедь, д. №430». На 1894 есть адрес лаврского кирпичного завода по «Низшая Лыбедь, 17», оборот – 40 000 рублей. Под другим данным, на тот же 1894, в «предместье Низшая Лыбедь», дом №17 – 1 000 000 кирпичей, оборот 16 000 рублей, 52 рабочих. 1910: 58 рабочих, 2 264 400 кирпичей, оборот 365 000 рублей, продано кирпича на 36 582 рубля 60 копеек. 1911: 50 рабочих. 1913: 2 000 000 кирпичей, 50 рабочих.

Про клеймо \textbf{КП число}, которое некоторые исследователи относят к казенным заводам. Почему я считаю, что КП это Киево-Печерская Лавра? На Замковой горе есть остатки кирпичной стены. Она была построена в качестве ограды кладбища Флоровского монастыря в 1855-1857 годах. Клейма на желтых и красных кирпичах стены: «КП I848», «КП I849», «КП I854», «КП I864» (причем я видел «КП I864» совершенно разные – и желтые, и алые, то есть произведенные из разной глины), «КП 87I8» (намеренно пишу первой «I», а не «1», чтобы вернее передать начертание). 

Как мы знаем, казенные заводы Инженерного департамента прекратили своё существование в 1861-м. Конечно, клеймо могло принадлежать и заводу Киевского приказа общественного призрения, однако использование клейма «КП I864» в стене, завершенной к 1857 году, наводит на мысль, что число означает не год, а нечто иное. Да и клеймо «КП 87I8» не может указывать на год выпуска.

Учитывая, что завод Лавры был одним из крупнейших производителей кирпича, клеймо «КП» встречается часто, я полагаю логичным соотносить клеймо «КП» с Киево-Печерской Лаврой.\\ 

\noindent\textbf{Киево-Печерская Лавра} – некий завод, им заведует Правление духовного собора Лавры. На 1900 год 15 пришлых рабочих. О каком заводе идет речь?\\

\noindent\textbf{Кирилловский монастырь} – владел заводом у юго-восточной части Лысой горы (современная Юрковица).

Время устроения завода мне точно неизвестно. Еще в 1747 году монастырь за 5 рублей арендовал завод софиевского эконома Варсанофия, но тогда же выделил 55 рублей кирпичных дел мастеру Якову Верченко, «на завод цегельный». Верченко запустил производство, получая по 10 рублей в месяц. 

В том же году монастырь продал в Петро-Павловскую обитель 32 100 кирпичей за 70 рублей 62 копейки, по 1,20 руб. за тысячу кирпичей, со скидкой и в рассрочку на несколько лет. Другому покупателю монастырь продал 34 150 кирпичей за 75,12 рублей (по 2,20 за тысячу штук).

В 1748 году продажа: 40 000 за 88 рублей, по цене 11 золотых за тысячу.

1749, продажа: отцу Филимону на 80 рублей, Михайловскому монастырю на 20, монахине Богословского монастыря на 2 рубля (1000 кирпичей).

1750, продажа: Братскому монастырю 10 000 кирпичей (8 золотых за тысячу), подольскому мещанину 5 000 за 8 рублей.

1765: священнику Флоровского монастыря Филипу, 15 000 кирпичей.

1779: произведено 215 000 обожженных кирпичей, сырца 107 400 (не знаю, шел ли он в счет обожженного или продавался отдельно), в монастырь перевезено 155 500, продано 109 700.

1780: произведено 135 000 обожженных кирпичей, сырца 198 800, перевезено в монастырь 146 850.

1782 год, в монастырь привезено 36 000 кирпичей, перевозка обошлась в 15,12 рублей, на этом документальные сведения о кирпичном заводе исчерпываются, вероятно после он был закрыт или перешел к другому владельцу.

В начале 1770-х Кирилловский монастырь иногда сам покупал кирпич, в 1772 году приобрел 20 000 кирпичей за 49,50 рублей.

На заводе работали мастера (отвечали за форму и обжиг) и рабочие, первые получали процент с готовых кирпичей, рабочие – за загрузку в печь сырца и выгрузку обожженного кирпича из печи.

Особый человек разыскивал пригодную глину и разрабатывал глинище. Плата за такой труд – 20 апреля 1761 года только за обнаружение горы с глиной заплачено 78 копеек, за расчистку глинища вскоре – 50 копеек.

Статистика дохода рабочих за выгрузку кирпича в 1779 года: 

\noindent 27 апреля, 13 200 кирпичей – 1,74 руб.\\
30 мая, 34 000 – 2,70 руб.\\
9 июня, 40 500 – 3,36 руб.\\
28 июня, 41 000 – 3,28 руб.\\
25 августа, 37 200 – 2,9 руб.\\
3 ноября, 33 000 – 3 руб.\\
14 ноября, 16 300 – 1,46 руб.\\

Статистика за загрузку в печь, тот же год:

\noindent 2 августа, 38 500 – 3,8 руб.\\
6 октября, 38 900 – 3,11 руб.\\
 
Зарплаты мастеров в примерах. В 1780 году Тымиш Кудинец и Омельян Бортник за 187 000 кирпичей получили 147,84 руб.

Отдельной статьей расхода была перевозка кирпичей в монастырь.\\

\noindent\textbf{Кирпичный завод цементного завода} – располагался на Юрковице. На 1900 год (и ближайшие годы) арендовался тюремным ведомством. В 1900-м на заводе работало 165 арестантов и было произведено 3 миллиона кирпичей. Вместо арендной платы, ведомство отдавало 1,2 миллиона кирпичей в год владельцу завода.

Поначалу арестантов водили на работу из тюрьмы, но затем построили бараки на 120 человек прямо на заводе. Арестанты были чернорабочими, а формовщиками работали вольнонаемные. На заводе трудилось также 38 лошадей, купленных тюремной инспекцией. Чтобы отказаться от лошадиного труда (для перевозки глины), позже проложили рельсы для вагонеток.

Каждый арестант получал, кроме «жилья» и харчей, 20 копеек в день, однако на руки только 30 процентов из них, то есть 6 копеек. 

На заводе была одна гофмановская печь, способная обжигать 22-24 тысячи кирпичей в сутки.\\


\noindent\textbf{Кирпичный завод} – 1926: Демиевка, «Голосеев., 35», тел 41-29.\\

%\textbf{Кирьянов И.А.} (1909) – «в Гавани».\\

%\textbf{Кирьянов И.Я.} – возможно, тот же самый Кирьянов И.А. На 1915 год у И.Я. с Чернояровым Л.Б. была общая контора на Набережно-Крещатицкой, 11 – стало быть «в Гавани».\\

\noindent\textbf{Козлов Сергей Яковлевич} – завод в Стайках. 1903: 30 рабочих.\\

\noindent\textbf{Крыжановский Гавриил} – завод в селе Дивин (Брусиловский район), известен в 1860-х.\\

\noindent\textbf{Куприенко Петр Максимович}, крестьянин – завод в Стайках. 1894: арендатор «Шлем. Дув. Лельчицкий».\\

\noindent\textbf{Кухаренко Антон Григорьевич}, крестьянин – завод на Юрковской, 4, с 1856 года. 1894: 300 000 кирпичей, оборот 3 000 рублей, 10 рабочих. 

Может это арендатор у Фиркс, ибо адрес Фиркс был Нижнеюрковская, 4, а чехарда с частями улицы Юрковской вносит путаницу. В адресной книге 1882 года по адресу Нижняя Юрковица, 4 – усадьба наследников Романовских (вот там и будет потом завод Фиркс), и на Юрковской, 4 – усадьба мещанина Семенченко.\\

\noindent\textbf{Латышев Яков Михайлович}, мещанин – завод по адресу Глубочицкая (Глубочицкое шоссе), 43, с 1886 года. 1894: 200 000 кирпичей, оборот 2 000 рублей, 6 рабочих. 1900: 5 местных рабочих. 1907: адрес прежний.\\

\noindent\textbf{Лейченко Аврум}\footnote{Аврум вместо Абрам, Дувим вместо Давид – идишские варианты танахических (библейских) имен. Некий рядовой Лейченко Абрам Янкелевич, иудей, из села Склопол Васильковского уезда пропал без вести 05/10-11/10 1914 года, числится в списках нижних чинов, погибших, раненых и пропавших без вести в 1-ю Мировую войну.} – завод в Василькове.\\

\noindent\textbf{Лонский А, Зелинский С.} – Демиевка, в 1907 году завод на Голосеевской, 2.\\ 

\noindent\textbf{Лурье и Левин} – этот завод известен мне по скудным сведениям. В 1899 где-то на «Верхней Юрковице» были начаты работы по устройству кирпичного завода, там же и был раскопан археологами Беляшевским и Скрыленко знаменитый Курган-Могикан. Адрес я отыскать не мог, однако речь идет о довольно большом участке горы. В справочнике 1882 года указана усадьба купца Соломона Лурье на улице Нижняя Юрковица, 37 – эдак за десяток домов от дома поручика Чеберяка. Но усадеб у Лурье могло быть несколько, и тот ли это Лурье? В деле Бейлиса упоминается еще «маслобойня Лурье».

В 1902 году производство на заводе Лурье и Левина «на Кирилловской улице» было приостановлено из-за кирпичного кризиса.\\

\noindent\textbf{Масленников Степан}, мещанин – 1900: 15 рабочих, из них четверо местных.\\ 

\noindent\textbf{Масс}, французский гражданин – завод в Новых Петровцах с 1883 года. 1887: 326 000 кирпичей, оборот 5 000 рублей, 14 рабочих.\\

\noindent\textbf{Меерович и Ко} – 1912: Юрковская, 7-9. А годом раньше, в 1911-м, по адресу Юрковская, 7, записана усадьба уже известного нам Иозефата Андржеиовского, которая могла быть парной к «нижней» его усадьбе на Кирилловской, где Андржеиовский держал кафельный завод. На тот же 1911 год, Юрковская 9 – усадьба мещанина «Севрюка Ан. Емел».\\

\noindent\textbf{Милевский Ф., Комарович М.} – середина 19 века, кирпичный завод в Григоровке (8 км от Триполья). В конце 18 века Милевский купил деревню у К. Браницкого.\\

\noindent\textbf{Минут и Ко}, – акционерное общество Минут и Ко, завод на Демиевке в 1910-15 годах. Адрес на 1910: Голосеевская, 30. 1911: Голосеевская, 5. 1915: Голосеевская, 35.\\

\noindent\textbf{Мозговой Федор Алексеевич}, купец – завод с 1870 года в Вышгороде. 1894: 500 000 кирпичей, оборот 8 000 рублей, рабочих 40.\\

\noindent\textbf{Наследники Дынникова} – завод на Нижне-Юрковской, 53. 1894: арендатор «Анас. Иван. Фокин».\\

\noindent\textbf{Наследники Молодова} – 1903: кафельный завод, Кирилловская 44, арендатор Давит Хургин.\\

\noindent\textbf{Немировский} – про него мне неизвестно ничего более кроме количества произведенных в 1900 году кирпичей – 4 миллиона штук.\\

\noindent\textbf{Новинский Ромуал Доминикович} – 1911: Кирилловская, 47; 52 рабочих.\\

\noindent\textbf{Общество крестьян с. Пирогова} – Пирогово. 1894: арендатор купец Афиноген Степанович Лунев.\\

\noindent\textbf{Остафьев} – завод в Мостыще (Мостище), основан в 1868 году. Есть село Мостище на реке Ирпень, а это было другое Мостище, оно теперь входит в состав Гостомеля как район, а Гостомель, в свою очередь – около Бучи, а Буча лежит возле города Ирпеня.\\

\noindent\textbf{Петровские кирпичные заводы} – по Сырцу (улицы Сырецкая и Копыловская) на основе бывших заводов Булышкиных, Марр, Михельсона и затем выше по течению Сырца. В 1944 году под карьеры отведен дополнительный земельный участок в Бабьем Яру. Выпуск в 1945 – 2 100 000 кирпичей, 1948 – 10 100 000, 1949 – 17 800 000, 1950 – 31 700 000. 

На 1939 год, сведения про завод: «Цегельный №1 Облбуду», адрес: Сырецкая 72, телефон: 3-25-71. Название на начало 21 века – Петровский завод стенных материалов и конструкций (Сырецкая, 33 – не столь далеко от Сырецкой 72, однако по другой стороне улицы).\\ 

\noindent\textbf{Реузов Федор Иванович} – 1903: Нижне-Юрковская, 6 (напомню адрес завода баронессы Фиркс – Нижнеюрковская 4, 6), 88 рабочих.\\

\noindent\textbf{Розенфельд Хаим}, купец – завод в селе Совки с 1881 года. 1887: 350 000 кирпичей, оборот 5 000 рублей, 14 рабочих.

Глинища заводов в Совках, вообще разных владельцев, я могу предположить на современной улице Кайсарова, по восточной ее стороне, на склоне горы, а также к северу от Совок, на северо-запад от улиц Яблоневой и Сигнальной, тоже на склонах.\\

\noindent\textbf{Романовский}, «киевский гражданин» – сведения об его заводе, вероятно по месту будущего завода баронессы Фиркс, находим в «Статистическом описании Киевской губернии»\cite{fundstat} 1852 года издания:

\begin{quotation}
Возле вышеописанного завода\footnote{Гудим-Левковичей.} у той же горы есть другой кирпично-кафельный завод Киевского гражданина Романовского. 

Он устроен в урочище, называемом Юрков-проток, на усадьбе, заключающей в себе 13 десятин земли. Этот завод существовал еще до 1816 года, в котором он приобретен настоящим владельцем. С того времени ежегодно, согласно с увеличением требования на кирпич и кафли, делались улучшения завода, а в 1837 году он совсем перестроен, на что, по показанию заводчика, издержано им более 6 000 рублей серебром.

В настоящее время завод состоит из следующих частей: деревянный корпус в 12 саженей длины, с обжигательными печами, для выделки и глазуровки кафель. Они обжигаются в особой простой печи под навесом, в 4 сажени длиною. Сарай для складки кафель на 7 саженях, сарай для дров, т.е. лучины для обжига кафель на 9 саженях; для делания кирпича два сарая, каждый по 12 сажень длиною; для складки вызженнаго кирпича сарай на 8 сажней; обжигательная печь на 25 000 кирпича укладки под навесом. Все эти строения деревянные, крытые драницей.

Глина и песок, употребляемые для кирпича и кафель, добываются в ущельи урочища Юркова-протока, принадлежащего к усадьбе заводчика\footnote{Ставок и текущий оттуда ручей были в овраге у западной границы нынешней военной части на Нижнеюрковской, которая занимает место срытого отрога Лысой горы, известной в наше время как Юрковица.}.

Для выделки глазурных белых изразцов материалы, а также и дрова для обжига кафель и кирпича, которых в год выходит 120 сажень, покупаются в Киеве, по таким же ценам, как и другими заводчиками.

[...] кирпич по свойству глины выходит слабый и употребляется для печей, продается за 6 рублей 50 копеек серебром за тысячу\footnote{К концу 19 века тысяча кирпичей стоила уже 28-30 рублей.}.

На кирпичном заводе работают две формы, то есть 4 человека, тоже издельно по 1 рубль 15 копеек серебром за выделку тысячи штук сырого кирпича, с поставкою под сарай.

Для нагрузки сырого кирпича в печь и выгрузки из нее нанимаются поденьщики, которые получают от тысячи за нагрузку по 30 копеек серебром, а за выгрузку по 15 копеек серебром. На подвозку глины, кирпич и кафли тоже употребляются поденьщики, с платою по 20 копеек серебром в день. Для обжигания кирпича нанимается особый мастер; ему за труб по 15 копеек серебром от тысячи.

Выделывают кирпич летом не более четырех с половиной месяцев и зарабатываются в это время 4 человека 172 рубля 50 копеек серебром; а обжигальщик получает в полтора месяца 22 рубля 50 копеек серебром. [...]

За 150 000 штук кирпича – 1 500 рублей.
\end{quotation}

Адрес завода Романовского вычислить относительно просто. В «Прибавлениях» к «Путеводителю» Бублика, за 1882 год, в списке усадеб по улице Нижняя Юрковица, значится: «Хныкина Алексея – 1, Гудим-Левковича – 2, Кошицовых насл. – 3, Романовских насл. 4».\\ 

\textbf{Рыбальский Степан} – в деле 1782 года «по предлож. е. с. гр. П. А. Румянцева-Задунайского, о шляхетском безчестии киевским мещанам, бывшим на уряде и с онаго перемененным, на подгородьи Преорке» содержится список хуторов Преорки и Куренёвки, где кроме прочего упомянут кирпичный завод на одном из хуторов:

\begin{quotation}
Хутор киевского мещанина Степана Рыбальского, заведенный козаком киевском Кондратом Чесноком, чему будет более восьмидесяти лет, после коего владел сын его, Чеснока, Николай, а от оного Николая достался по купле ему, Рыбальскому, в прошлом 1757 году декабря 31 дня. В оном хуторе строение: две избы, с коих одна с комнатою, и амбар под одною кровлею, изба вновь выстроенная, одна изба людская, винокурня и в ней хатка для винокура, сажалка, с которой проведена вода в винокурню, и при оном огород и сад, кирпичный завод. Под оными строениями, кирпичным заводом и огородом в окружности меры 1338 аршин. В год взымается окладу от аршина по полушке 3 р. 34 1/4 коп. Близ оного хутора роща меры 1915 аршин.
\end{quotation}

\noindent\textbf{Рыкун} – еврейская земледельческая колония под Дымером (Вышгородский район), в 1887 имела три печи для обжига кирпичей. Ныне Рыкун входит в состав Дымера.\\

\noindent\textbf{Свято-Троицкий монастырь} – вероятно, Свято-Троиц\-кий больничный монастырь Лавры, а не Ионинский. 1894: завод в Корчеватом; арендаторы купец Яков Николаевич Бернер, Наследники купца Адама Снежко, Иларион Алек. Фокин.\\

\noindent\textbf{Серебренников (Серебряников)} – купец, в 1879-82 годах владел заводом на Сырецкой 30, 31, западнее заводов Рихерта-старшего, выше по течению речки Сырец, по левой ее стороне.

Возможно, Серебренников и Серебряников – разные люди, потому что в справочнике 1882 года, в списке владельцев участков по Сырецкой, рядом с усадьбой номер 35 стоит «Серебренников граж.» (и чернилами пометка «среб»), а «Серебренников» по номерам усадеб 30 и 31 чернилами переправлен в «Серебряников» и дописано «купец».\\

\noindent\textbf{Скворцов Федор Степанович} – 1894: Байкова гора, арендатор «Хаим. Арон. Шутый».\\

\noindent\textbf{Скворцова Марфа Осиповна}, крестьянка – 1894: Байковая гора, №119. Завод основан в 1873 году.\\

\noindent\textbf{Скомаровский Мордух}, мещанин – завод в Витачево с 1881 года. 1887: 190 000 кирпичей, оборот 2 000 рублей, 18 рабочих.\\

\noindent\textbf{Снежко Сергей Адамович} – 1903: Корчеватое, 98 рабочих.\\

\noindent\textbf{Софийский митрополичий дом} (Софиевский собор) – Простасов яр, 46. Завод основан в 1840 году. 1894: \mbox{300 000} кирпичей, оборот 3 000 рублей, 21 рабочий, арендатор «австрийский под. Ант. Вас. Панст.».

Историю кирпичных заводов Софии я по возможности проследил с 18 века. В деле Киевской городской канцелярии 1752 года «о пойманных польским ротмистром Рокицким двух разбойниках Григорие Киселенке да Григорие Пархоменке» приведены показания Ивана Сергеева, по прозвищу Кошин и Гончар, 27 лет. 

\begin{quotation}
он, Кошин,  после отца и матери обретался в оных же кожемяках у тамошнего жителя Ивана Плесуна, в гончарской науке лет 10-ть и более, а потом находился близ крещатской пристани, ведомства Киево-софийской катедры\footnote{Собора.} для делания кирпичей, где будучи на приставленной у оных кирпичей старостою Ивана Кучера дочери Агафии одружился; 

и в прошлом 1751 году, когда оный тесть его из оного Крещатика перешел на Лыбедь, ведомства Киево-софийской катедры к цегельням, для делания кирпича, то тогда и он с ним перешел [...]
\end{quotation} 

Из сего заключаю, что речь идет о двух кирпичных заводах. Один – на тогдашнем Крещатике, местности на склоне около нынешней лестницы с памятником Магдебургскому праву. Крещацкая пристань – это у Почтовой площади. А цегельни на Лыбеди – вероятно цегельни в Протасовом яру, ведь мимо него протекает Лыбедь.\\

\noindent\textbf{Фокин Ефим}, купец – 1900: 27 рабочих (5 местных).\\

\noindent\textbf{Фузик Степан Макарович (Маркович)} , крестьянин – Шулявка, Борщаговская, 126. На 1894, по заводу 1881 года основания: 550 000 кирпичей, оборот 6 000 рублей, 15 рабочих. На 1894, по заводу 1875 года основания: 300 000 кирпичей, оборот 3 000 рублей, 10 рабочих. Оба завода по одному адресу. Сколько же было заводов?

Корни завода или заводов тянутся, вероятно, к братьям Фузикам – Семену и Виктору – крестьянам из Беличей. На Шулявке, около Нивок, братья основали хутор Фузиковщину да кирпичный завод, и судя по расположению это завод тот же, что у Фузика Степана.

На 1907 есть данные «Фузик Борщагов. 12», но тот ли это Фузик?\\

\noindent\textbf{Халемский Моисей}, купец – завод в Новых Петровцах с 1878 года. 1894: 652 000 кирпичей, оборот 7 000 рублей, 30 рабочих.\\

\noindent\textbf{Хелемская Сура Вольфовна} – Новые Петровцы. 1903: 80 рабочих.\\

\noindent\textbf{Хлебникова Елена Адамовна, Фокина Марфа Кирилловна} – 1903: завод в Корчеватом, 100 рабочих. Хлебникова – дочь Адама Снежко. Фокина Мария (Марфа) имела отдельное клеймо, см. «Клейма».\\

\noindent\textbf{Цегельный Петр и Григорий}, крестьяне – завод в Василькове с 1872 года, располагался возле нынешнего адреса Цегельный переулок, 1. Название проулка часто переводят в «Кирпичный», забыв о фамилии Цегельных. Местность тамошняя именуется Лысой горой.

На 1887, выпуск 150 000 кирпичей, оборот 3000 рублей, 8 рабочих.\\

\noindent\textbf{Чернов}, купец – в середине 19 века его завод был на ручье Коноплянке (Кониченке), на территории ныншнего интерната на Бережанской, 6.\\

\noindent\textbf{Чирков (Чарков, Черков, Чернов) Харитон Фадеевич}, крестьянин, старообрядец – завод на Байковой горе с 1879 года. На 1887: 300 000 кирпичей, оборот 3 000 рублей, 10 рабочих. По другим данным, на 1887: 300 000 кирпичей, оборот 4 000, 11 рабочих. Так что возможно это два разных завода, близких по местности. Похилевич пишет в книге 1887 года, что завод «Чернова» оценивается в 300 рублей. По сведениям 1894 года указано лишь село Совки, где завод, основанный в 1881, давал 540 000 кирпичей, оборот 6 000 рублей, 18 рабочих.

На 1900, есть данные о «трех кирпичных заводах» в деревне Совки, принадлежащих Харитону Чиркову, Федору Скворцову и Димитрию Сингаевскому. На каждом заводе работало по 16 рабочих с Черниговщины.

Кроме того, в деревне Совке, в усадьбе Харитона Чиркова находилось старообрядческое кладбище, по сведениям полиции – «с незапамятных времен».\\

\noindent\textbf{Черков Кл. Хар.} – сын Чиркова Харитона, сведения на 1915 год просты: «Байкова гора. Продажа всех сортов кирпича по самым умеренным ценам».\\

\noindent\textbf{Шаврова София Викторовна} – на 1911: завод по адресу Сырецкая, 25, 130 рабочих. Однако в 1909, 1912, 1913: указан адрес Большая Васильковская, 139. На 1911 адрес: Малая Благовещенская 130, Большая Васильковская 139. 1913: арендатор завода на Большой Васильковской: Гитес А. А.

На карте 1914 года завод показан между заводами Бернера и Журавлева, на нынешней Лыбедской площади, условно говоря между Домом с тарелкой и Домом музыки: 50.41190985356343, 30.52568002137533\\

\noindent\textbf{Шевченко С.К.} – 1912: Кирилловская, 47-49.\\ 

\noindent\textbf{Шембек И. А.}, граф – на 1913 год известен его завод в урочище Дубинец под Бородянкой, управляющий Савранский Ф. Л., производство 2 000 000 кирпичей, 22 рабочих.\\

\noindent\textbf{Шехтель Иосиф Иванович}, инженер-техник – завод в селе Халепье с 1881 года. 1887: 280 000 кирпичей, оборот 5 000 рублей, 21 рабочий.\\

\noindent\textbf{Шульгин Ферапонт}, купец – завод с 1886 года на Глубочицкой улице, 30. 1894: 200 000 кирпичей, оборот \mbox{2 000} рублей, 6 рабочих. 1900: 8 местных рабочих. 1907: адрес указан Глубочицкое шоссе, 40. Вместо «Шульгин» в одном источнике стоит «Шукагин» при заводе по номеру 30 за тот же 1907, так что думаю завод Шульгина никуда с 30 номера не перебирался.\\

\noindent\textbf{Шуминская Е.Н. и Ко.} – на 1907-й адрес указан «Александров. 31», а это Подол, близ пристани, стало быть контора. Где был завод, я не знаю.\\

\noindent\textbf{Ясногурский Филипп Норбертович} (1841-1913), купец?, коллежский асессор?, гласный думы – завод с 1875 года указан по адресу «Глубочица, 16». 

1887: \mbox{300 000} кирпичей, оборот 3 000 рублей, 15 рабочих.

Ясногурский был также одним из учредителей Киевского Союза Русского Народа (СРН), публицистом, членом Киевского отдела Российского Общества покровительства животным, Лукьяновского Попечительства о бедных. В 1909  году Лукьяновский отдел СНР открыл в доме Ясногурского на Монастырской улице дневной приют для детей. Ясногурский владел на Глубочицкой довольно большой усадьбой, и устроил через нее улицы – теперь это улицы Пимоненко, Дончука (бывшая Филипповская), да Филипповский переулок. На отдельном участке усадьбы Ясногурский устраивал зимой платный каток, в 1912 году там же спортсмены взяли землю в аренду и соорудили стадион, где в советское время построили обувную фабрику – ныне на том месте бизнес-центр.\\

\noindent\textbf{Ярошинский К. И.} – Карл Иосифович Ярошинский, он же Кароль Юзефович (1878-1929), на 1915 год числится владельцем бывшего завода Субботиной по Большой Васильковской, 139. 

Ярошинский был крупным землевладельцем, промышленником и финансовым воротилой.\\